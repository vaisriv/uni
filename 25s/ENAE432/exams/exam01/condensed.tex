\documentclass[8pt]{extarticle}

\usepackage[utf8]{inputenc}
\usepackage{amsmath}
\usepackage{amssymb}
\usepackage{multicol}
\usepackage{blindtext}
\usepackage{geometry}
\geometry{a4paper, left=5mm, right=5mm, top=5mm, bottom=12mm}
\setlength{\parindent}{0pt}
\setlength{\columnsep}{0.4cm}
\setlength{\columnseprule}{0.5pt}
\usepackage{enumitem}
\usepackage{svg}
\usepackage{dsfont}
\usepackage{hyperref}

\usepackage{fancyhdr}
\usepackage{extramarks}
\usepackage{amsthm}
\usepackage{amsfonts}
\usepackage{xparse}
\usepackage{bm}
\usepackage{tikz}
\usepackage{graphicx}
\usepackage{caption}
\usepackage{subcaption}
\usepackage[plain]{algorithm}
\usepackage{algpseudocode}
\usepackage{listings}
\usepackage[per-mode = fraction]{siunitx}
\usepackage{calc}
\usepackage{cancel}
\usepackage{calligra}

\usepackage{xcolor}
\definecolor{myblue}{cmyk}{1,.72,0,.38}
\definecolor{mygreen}{RGB}{78, 153, 67}     % A vibrant green
\definecolor{myred}{RGB}{220, 53, 69}       % A bright red
\definecolor{myyellow}{cmyk}{0,0,1,0}       % Pure yellow
\definecolor{mydarkblue}{cmyk}{1,.6,0,.4}   % A darker shade of blue
\definecolor{mypurple}{RGB}{104, 33, 122}   % A deep purple
\definecolor{myorange}{RGB}{253, 126, 20}   % A bright orange
\definecolor{mygray}{gray}{0.6}             % A medium gray, 0 is black, 1 is white
\definecolor{myteal}{RGB}{0,128,128}        % A teal color

%
% Various Helper Commands
%

% Such That
\newcommand{\st}{\text{ s.t. }}

% Useful for algorithms
\newcommand{\alg}[1]{\textsc{\bfseries \footnotesize #1}}
\NewDocumentCommand{\func}{m o}{\mathrm{#1}\IfValueT{#2}{(#2)}}

% For derivatives
\newcommand{\deriv}[2]{\frac{\mathrm{d}}{\mathrm{d}#1}\left(#2\right)}
\newcommand{\derivsec}[2]{\frac{\mathrm{d}^{2}}{\mathrm{d}{#1}^{2}}\left(#2\right)}

% For partial derivatives
\newcommand{\pderiv}[2]{\frac{\partial}{\partial#1}\left(#2\right)}
\newcommand{\pderivsec}[2]{\frac{\partial^{2}}{\partial{}{#1}^2}\left(#2\right)}

% Probability commands: Expectation, Variance, Covariance, Bias
\NewDocumentCommand{\E}{o}{\mathrm{E}\IfValueT{#1}{(#1)}}
\NewDocumentCommand{\Var}{o}{\mathrm{Var}\IfValueT{#1}{(#1)}}
\NewDocumentCommand{\Cov}{o}{\mathrm{Cov}\IfValueT{#1}{(#1)}}
\NewDocumentCommand{\Bias}{o}{\mathrm{Bias}\IfValueT{#1}{(#1)}}
\NewDocumentCommand{\Prob}{o}{\mathrm{Pr}\IfValueT{#1}{(#1)}}

% Col and Row Vectors
\newcommand{\crvector}[1]{\ensuremath{\begin{pmatrix}#1\end{pmatrix}}}

% For writing vectors
\let\oldhat\hat{}
\let\oldvec\vec{}
\renewcommand{\vec}[1]{\oldvec{\mathbf{#1}}}
\newcommand{\vecb}[1]{\mathbf{#1}}
\renewcommand{\hat}[1]{\oldhat{\mathbf{#1}}}

\newcommand{\cvect}[2]{ \begin{pmatrix} #1 \\ #2 \end{pmatrix} }
\newcommand{\ctvect}[3]{ \begin{pmatrix} #1 \\ #2 \\ #3 \end{pmatrix} }
\newcommand{\vect}[2]{ \langle{} #1, #2 \rangle{} }
\newcommand{\tvect}[3]{ \langle{} #1, #2, #3 \rangle{} }
\newcommand{\qvect}[4]{ \langle{} #1, #2, #3 \rangle{} }

% For sin and cos and tangent, etc
\let\oldsin\sin{}
\renewcommand{\sin}[1]{\oldsin(#1)}
\let\oldcos\cos{}
\renewcommand{\cos}[1]{\oldcos(#1)}
\let\oldtan\tan{}
\renewcommand{\tan}[1]{\oldtan(#1)}
\let\oldcsc\csc{}
\renewcommand{\csc}[1]{\oldcsc(#1)}
\let\oldsec\sec{}
\renewcommand{\sec}[1]{\oldsec(#1)}
\let\oldcot\cot{}
\renewcommand{\cot}[1]{\oldcot(#1)}
\let\oldsinh\sinh{}
\renewcommand{\sinh}[1]{\oldsinh(#1)}
\let\oldcosh\cosh{}
\renewcommand{\cosh}[1]{\oldcosh(#1)}
\let\oldtanh\tanh{}
\renewcommand{\tanh}[1]{\oldtanh(#1)}
\let\oldcoth\coth{}
\renewcommand{\coth}[1]{\oldcoth(#1)}
\let\oldarcsin\arcsin{}
\renewcommand{\arcsin}[1]{\oldarcsin(#1)}
\let\oldarccos\arccos{}
\renewcommand{\arccos}[1]{\oldarccos(#1)}
\let\oldarctan\arctan{}
\renewcommand{\arctan}[1]{\oldarctan(#1)}

% For log, ln, etc
\let\oldlog\log{}
\renewcommand{\log}[1]{\oldlog(#1)}
\let\oldln\ln{}
\renewcommand{\ln}[1]{\oldln(#1)}

% For Re, Im, etc
\let\oldRe\Re{}
\renewcommand{\Re}[1]{\oldRe\{#1\}}
\let\oldIm\Im{}
\renewcommand{\Im}[1]{\oldIm\{#1\}}

% For Laplace, ILaplace, etc
\newcommand{\laplace}[1]{\mathscr{L}\{#1\}}
\newcommand{\ilaplace}[1]{\mathscr{L}^{-1}\{#1\}}

% For integrals
\let\oldint\int{}
\NewDocumentCommand{\intidef}{m o}{\oldint{} #1 \IfValueT{#2}{\mathrm{d}#2}}
\newcommand{\intdef}[4]{\oldint_{#1}^{#2} #3 \mathrm{d}#4}

% Defining Units that are not in the SI base
\DeclareSIUnit\bar{bar}
\DeclareSIUnit\foot{ft}
\DeclareSIUnit\inch{in}
\DeclareSIUnit\year{yr}
\DeclareSIUnit\hour{hr}
\DeclareSIUnit\failure{failure}
\DeclareSIUnit\cycle{cycle}
\DeclareSIUnit\DU{DU}
\DeclareSIUnit\AU{AU}
\DeclareSIUnit\TU{TU}
\DeclareSIUnit\dollar{\$}
\DeclareSIUnit\cent{\text{\textcent}}
\DeclareSIUnit\c{\degreeCelsius}

% Griffiths script char LOL
\DeclareMathAlphabet{\mathcalligra}{T1}{calligra}{m}{n}
\DeclareFontShape{T1}{calligra}{m}{n}{<->s*[2.2]callig15}{}
\newcommand{\rcurs}{\mathcalligra{r}\,}
\newcommand{\brcurs}{\pmb{\mathcalligra{r}}\,}

% Code Listing config
\definecolor{codegreen}{rgb}{0,0.6,0}
\definecolor{codegray}{rgb}{0.5,0.5,0.5}
\definecolor{codepurple}{rgb}{0.58,0,0.82}
\definecolor{backcolour}{rgb}{0.95,0.95,0.92}
\lstdefinestyle{overleaf}{
	% backgroundcolor=\color{backcolour},
	commentstyle=\color{codegreen},
	keywordstyle=\color{magenta},
	numberstyle=\tiny\color{codegray},
	stringstyle=\color{codepurple},
	basicstyle=\ttfamily\footnotesize,
	breakatwhitespace=false,
	breaklines=true,
	captionpos=b,
	keepspaces=true,
	numbers=left,
	numbersep=5pt,
	showspaces=false,
	showstringspaces=false,
	showtabs=false,
	tabsize=4
}

% \usepackage[latte]{catppuccinpalette}
% \lstdefinestyle{catppuccin}{
% 	breaklines=true,
% 	keepspaces=true,
% 	numbers=left,
% 	numbersep=5pt,
% 	showspaces=false,
% 	showstringspaces=false,
% 	breakatwhitespace=true,
% 	tabsize=4,
% 	stringstyle = {\color{CtpGreen}},
% 	commentstyle={\color{CtpOverlay1}},
% 	basicstyle = {\small\color{CtpText}\ttfamily},
% 	keywordstyle = {\color{CtpMauve}},
% 	keywordstyle = [2]{\color{CtpBlue}},
% 	keywordstyle = [3]{\color{CtpYellow}},
% 	keywordstyle = [4]{\color{CtpLavender}},
% 	keywordstyle = [5]{\color{CtpPeach}},
% 	keywordstyle = [6]{\color{CtpTeal}}
% }

\lstset{style=overleaf}

% TikZ settings for diagrams
\usepackage{tikz}
\usetikzlibrary{shapes,positioning,arrows,fit,calc,graphs,graphs.standard}

% Font and typography settings
\usepackage[T1]{fontenc}
\usepackage[nosf]{kpfonts} % Keeping kpfonts for mathematical symbols
\usepackage{sourcesanspro} % Sans-serif font for clarity

% Microtype for improved typography
\usepackage{microtype}

% Apply color to all math environments
% \everymath{\color{myblue}}
% \everydisplay{\color{myblue}}

% Customized spacing
\renewcommand{\baselinestretch}{.8}

% Page footers
\pagestyle{fancy}
\lfoot{\fsheetCourse: \fsheetExam}
\rfoot{\fsheetAuthor}

% Layer declaration for TikZ drawings
\pgfdeclarelayer{background}
\pgfsetlayers{background,main}

\usepackage{titlesec} % For customizing section titles
% Redefining section titles for a more compact look
\titleformat{\section}
  {\normalfont\large\sffamily\bfseries\color{myblue}}
  {\thesection}{1em}{}
\titleformat{\subsection}
  {\normalfont\normalsize\sffamily\bfseries\color{mygreen}}
  {\thesubsection}{1em}{}
\titleformat{\subsubsection}
  {\normalfont\normalsize\sffamily\bfseries\color{mypurple}}
  {\thesubsubsection}{1em}{}

% Adjust spacing before and after section titles
\titlespacing*{\section}{0pt}{1ex plus 1ex minus .2ex}{0.5ex plus .2ex}
\titlespacing*{\subsection}{0pt}{1ex plus 1ex minus .2ex}{0.5ex plus .2ex}
\titlespacing*{\subsubsection}{0pt}{1ex plus 1ex minus .2ex}{0.5ex plus .2ex}

% Adjust spacing for enumerate environment to make it more compact
\setlist[enumerate]{
    topsep=0pt, % Space between the first item and the preceding paragraph
    partopsep=0pt, % Extra space added to topsep when environment starts a new paragraph
    parsep=0pt, % Space between paragraphs within an item
    itemsep=1pt, % Space between items
    leftmargin=* % Adjusts the left margin to align with the surrounding text
}
\setlist[itemize]{
    topsep=0pt, % Space between the first item and the preceding paragraph
    partopsep=0pt, % Extra space added to topsep when environment starts a new paragraph
    parsep=0pt, % Space between paragraphs within an item
    itemsep=1pt, % Space between items
    leftmargin=* % Adjusts the left margin to align with the surrounding text
}

% Reduce spacing around all display math, affecting align environments
\AtBeginDocument{
  \setlength{\abovedisplayskip}{0pt} % Space above display math
  \setlength{\abovedisplayshortskip}{0pt} % Space above display math when preceding text line is short
  \setlength{\belowdisplayskip}{0pt} % Space below display math
  \setlength{\belowdisplayshortskip}{0pt} % Space below display math when following text line is short
}


\newcommand{\fsheetCourse}{ENAE432 - 0101} % chktex 8
\newcommand{\fsheetExam}{Exam01}
\newcommand{\fsheetAuthor}{Vai Srivastava}

\begin{document}
\begin{multicols*}{3}

\section*{Input-Output}
\begin{align*}
    \func{\ddot{y}}[t] &= K \func{u}[t], K = \frac{K_{f}K_{m}}{m} \\
    \func{y}[t] &= \intdef{0}{t}{\func{g}[t-\tau]\func{u}[\tau]}{\tau} \\
    \func{y}[t] &= \alpha_{1}\func{y_{1}}[t]+\alpha_{2}\func{y_{2}}[t]+\cdots+\alpha_{n}\func{y_{n}}[t]
\end{align*}

\section*{Complex Exponents}
\subsection*{General Form}
\begin{align*}
    \func{z}[t] &= \func{a}[t]+\func{b}[t]j \\
                &= \func{r}[t]e^{j\func{\theta}[t]} \\
    \sigma &: \text{amplitude envelope} \\
    \omega &: \text{oscillation frequency} \\
    s &: \text{complex frequency} \\
    r = |A| &: \text{initial amplitude} \\
    \phi = \angle A &: \text{phase shift} \\
    \phi > 0 &: \text{phase lead} \\
    \phi < 0 &: \text{phase lag}
\end{align*}

\subsection*{Basic Example}
\begin{align*}
    \func{z}[t] &= e^{st}, \quad s \in \mathbb{C} \\
    s &= \sigma + \omega j, \quad \sigma, \theta \in \mathbb{R} \\
    \Re{s} &= \sigma \\
    \Im{s} &= \omega \\
    \Re{e^{st}} &= e^{\sigma t} \cos{\omega t} \\
    \Im{e^{st}} &= e^{\sigma t} \sin{\omega t} \\
    e^{st} &= \begin{cases}
        e^{\sigma t} & \omega = 0 \\
        e^{j \omega t} = \cos{\omega t} + j \sin{\omega t} & \sigma = 0 \\
        e^{\sigma t} \left[\cos{\omega t}+j\sin{\omega t}\right] & \text{otherwise}
    \end{cases}
\end{align*}

\subsection*{Specific Example}
\begin{align*}
    \func{z}[t] &= Ae^{st}, \quad A, s \in \mathbb{C} \\
    s &= \sigma + \omega j, \quad \sigma, \theta \in \mathbb{R} \\
    A &= r e^{j \phi} \\
    Ae^{st} &= r e^{\sigma t} \left[\cos{\omega t + \phi}+j\sin{\omega t} + \phi\right] \\
    \Re{Ae^{st}} &= r e^{\sigma t} \cos{\omega t + \phi} \\
    \Im{Ae^{st}} &= r e^{\sigma t} \sin{\omega t + \phi}
\end{align*}

\section*{Transfer Function}
\begin{align*}
    \func{G}[s] &= \frac{\func{q}[s]}{\func{r}[s]} \\
    \func{q}[s] &= \laplace{\func{y}[t]} = \beta_m \prod_{i=1}^{m}(s-z_i) \\
    \func{r}[s] &= \laplace{\func{u}[t]} = \alpha_n \prod_{k=1}^{n}(s-p_k)
\end{align*}
\begin{enumerate}
    \item Get information on modes from homogenous response
    \item Get information on forced response from evaluating \( \func{G}[s] \) at specific values of \( s \)
\end{enumerate}

\subsection*{ZPK Form}
\[
    \func{G}[s] = K \left[\frac{\prod_{i=1}^{m}(s-z_i)}{\prod_{k=1}^{n}(s-p_k)}\right]
\]
\begin{enumerate}
    \item Zeroes: \( z_i \) satisfy \( \func{q}[z_i] = 0 \)
    \item Poles: \( p_k \) satisfy \( \func{r}[p_k] = 0 \)
    \item Gain: \( K = \frac{\beta_m}{\alpha_n} \) is always real
\end{enumerate}

\section*{Characteristic Polynomial}
\begin{align*}
    y &: \text{polynomial response} \\
    y_h &: \text{homogenous response} \\
    y_f &: \text{forced response} \\
    r &: \text{characteristic polynomial} \\
    p_i &: \text{roots of polynomial} \\
    n &: \text{\# of roots} \\
    l &: \text{\# of times roots are repeated} \\
    \func{r}[s] &= {\left(s-p_1\right)}^l\left(s-p_{l+1}\right)\cdots\left(s-p_n\right) \\
    \func{y_h}[t] &= (C_1 + C_2 t + \cdots + C_l t^{l-1}) e^{p_1 t} + \sum_{k = l+1}^{n} C_k e^{p_k t} \\
    \func{y}[t] &= \func{y_h}[t] + \func{y_f}[t]
\end{align*}

\section*{Modes}
\begin{enumerate}
    \item Solutions which are possible without any input
    \item Terms in solution for \( \func{y}[t] \) of form \( e^{pt} \), where \( r(p) = 0 \)
    \item First Order when \( p \in \mathbb{R} \)
    \item Second Order when \( p \in \mathbb{C} \)
\end{enumerate}

\subsection*{Stability}
\begin{enumerate}
    \item Mode is stable if: \( |e^{pt}| \to 0 \text{ as } t \to \infty \implies \sigma < 0 \) (root \( p \) lies in left half of complex plane)
    \item System is stable if: all modes are stable \( \implies \Re{p_k} < 0 \forall k \in \{1, \dots, n\} \)
    \item If the system is stable, \( \func{y_h}[t] = 0 \) for all initial conditions
    \item Repeated modes retain the stability of their roots
    \item For constant input, \( \func{y_{\text{tr}}}[t] = \func{y_h}[t] \) and \( \func{y_{\text{ss}}}[t] = \func{y_f}[t] \)
\end{enumerate}

\subsection*{Instability}
\begin{enumerate}
    \item Mode is unstable if: \( \sigma > 0 \) (root \( p \) lies in left half of complex plane)
    \item System is unstable if: any mode is unstable \( \implies \Re{p_k} > 0 \text{ for any } k \in \{1, \dots, n\} \)
\end{enumerate}

\subsection*{Marginal Stability}
\begin{enumerate}
    \item Mode is marginally stable if: \( \sigma = 0 \) (root \( p \) lies on complex axis)
    \item Repeated marginally stable modes will increase polynomially with t
\end{enumerate}

\subsection*{Transience}
\begin{enumerate}
    \item Transient Response: \( \func{y_{\text{tr}}}[t] \)
    \item Terms in \( \func{y}[t] \) for which: \( \lim_{t \to \infty} | \func{y_{\text{tr}}}[t] | \to 0 \)
    \item If the system is stable, \( \func{y_{\text{tr}}}[t] \) contains all of \( \func{y_h}[t] \) and any decaying terms of \( \func{y_f}[t] \)
\end{enumerate}

\subsection*{Steady-State}
\begin{enumerate}
    \item Steady State Response: \( \func{y_{\text{ss}}}[t] \)
    \item All other terms in \( \func{y}[t] \)
    \item Contains all marginally stable terms of \( \func{y_h}[t] \)
\end{enumerate}

\subsection*{Convergence}
\begin{enumerate}
    \item Quantifies how quickly stable modes decay to 0
    \item \qty{2}{\percent} Criterion defines the settling time:
        \[
            t_s \st |e^{pt}|\leq 0.02 \forall t \geq t_s
        \]
    \item For first order modes, \( t_s = \frac{\ln{0.02}}{\sigma} \approx \frac{4}{|\sigma|} \)
    \item Above approximation is a good tool for second order modes, but is less accurate due to oscillations
    \item Doubling time applies to unstable modes:
        \[
            |e^{\sigma t_d}| = 2 \implies t_d \approx \frac{0.7}{\sigma}
        \]
    \item Smaller doubling time \( \iff \) ``more unstable'' system \( \implies \) faster rate of increase in amplitude
    \item Settling times decrease the further left of the imaginary axis \( p \) is
    \item Doubling times decrease the further right of the imaginary axis \( p \) is
\end{enumerate}

\subsection*{Damping Ratio}
\begin{align*}
    \omega_n &= \sqrt{\sigma^2 + \omega_d^2} = |p_1| \\
    \zeta &= \frac{|\sigma|}{\omega_n} \\
    \zeta &= \begin{cases}
        0 \leq \zeta \leq 1 & \text{for stable modes} \\
        \zeta \approx 0 & \text{many oscillations before settled} \\
        \zeta \approx 1 & \text{less that one complete oscillation} \\
    \end{cases}
\end{align*}

\section*{Laplace Transform}
\subsection*{Definition}
\begin{align*}
    \func{f}[t] &= \frac{1}{2 \pi j} \intidef{\func{F}[s] e^{st}}[s] \\
    \func{F}[s] &= \intdef{0}{\infty}{\func{f}[t] e^{-st}}{t}
\end{align*}

\subsection*{Special Cases}
\begin{align*}
    \laplace{e^{pt}} &= \frac{1}{s-p} \forall p \in \mathbb{C} \\
    \laplace{Ae^{at}\cos{bt+\psi}} &= \frac{C}{s-p}+\frac{\overline{C}}{s-\overline{p}}, C = \frac{A}{2}e^{j\psi} \\
    \laplace{c} &= \frac{c}{s} \forall c \in \mathbb{C}
\end{align*}

\subsection*{Properties}
\begin{align*}
    \laplace{\func{f_1}[t] + \func{f_2}[t]} &= \func{F_1}[s]+\func{F_2}[s] \\
    \laplace{\func{f_1}[t]\func{f_2}[t]} &\neq \func{F_1}[s]\func{F_2}[s] \\
    \laplace{\func{\dot{f}}[t]} &= s \func{F}[s] - \func{f}[0] \\
    \laplace{\func{f^{(k)}}[t]} &= s^{k} \func{F}[s] - \sum_{i=1}^{k} s^{i-1}\func{f^{(k-i)}}[0] \\
    \laplace{t \func{f}[t]} &= -\deriv{s}{\func{F}[s]} \\
    \laplace{t e^{pt}} &= -\deriv{s}{\frac{1}{s-p}} \\
    \laplace{t^{k} e^{pt}} &= \frac{k!}{{(s-p)}^{k+1}}
\end{align*}

\subsection*{Usage}
\begin{align*}
    \func{r}[s]\func{Y}[s]-\func{c}[s] &= \func{q}[s]\func{U}[s]-\func{b}[s] \\
    \func{r}[s] &= \alpha_n s^n + \cdots + \alpha_1 s + \alpha_0 \\
    \func{q}[s] &= \beta_m s^m + \cdots + \beta_1 s + \beta_0 \\
    \func{c}[s] &= n-1 \text{ order poly from IC on } \func{y}[t] \\
    \func{b}[s] &= m-1 \text{ order poly from IC on } \func{u}[t] \\
    \func{Y}[s] &= \func{G}[s]\func{U}[s]+\left[\frac{\func{c}[s]-\func{b}[s]}{\func{r}[s]}\right]
\end{align*}

\section*{Inverse Laplace}
\subsection*{Partial Fraction Expansion}
\begin{align*}
    \func{Y}[s] &= \func{G}[s]\func{U}[s]+\left[\frac{\func{c}[s]-\func{b}[s]}{\func{r}[s]}\right] \\
                &= \left[\frac{\func{q}[s]}{\func{r}[s]}\right]\left[\frac{\func{a}[s]}{\func{h}[s]}\right]+\left[\frac{\func{c}[s]-\func{b}[s]}{\func{r}[s]}\right] \\
                &= \frac{\func{q}[s]\func{a}[s]+\func{h}[s]\left[\func{c}[s]-\func{b}[s]\right]}{\func{r}[s]\func{h}[s]} \\
                &= \sum_{l=1}^{L}\frac{A_l}{s-d_l} \\
    \func{y}[t] &= \sum_{l=1}^{L}A_l e^{d_l t}
\end{align*}

\subsection*{Residue Formula}
\begin{align*}
    A_l &= {\left[(s-d_l)\func{Y}[s]\right]}_{s=d_l} \\
    \overline{A_l} &= {\left[(s-\overline{d_l})\func{Y}[s]\right]}_{s=\overline{d_l}} \\
    A_l e^{d_l t} + \overline{A_l} e^{\overline{d_l} t} &= 2 \left|A_l\right| e^{\sigma t} \cos{\omega t + \angle A_l}
\end{align*}

\subsection*{Repeated Roots}
\begin{align*}
    L &: \text{\# of roots} \\
    K &: \text{\# times a root is repeated} \\
    \func{Y}[s] &= \sum_{l=1}^{K} \frac{A_l}{{(s-d_l)}^l}+\sum_{l=K+1}^{L}\frac{A_l}{(s-d_l)} \\
    \func{y}[t] &= \sum_{l=1}^{K}\left(\frac{A_l t^{l-1}}{(l-1)!}\right)e^{d_1 t} + \sum_{l=K+1}^{L}A_l e^{d_l t}
\end{align*}

\section*{State Model}
\begin{align*}
    \func{\dot{\underline{x}}}[t] &= A \func{\underline{x}}[t] + B \func{u}[t] \\
    \func{y}[t] &= C \func{\underline{x}}[t] + D \func{u}[t] \\
    \func{\underline{x}}[t] &= \crvector{\func{x_1}[t] \\ \vdots \\ \func{x_n}[t]},\quad B = \crvector{b_1 \\ \vdots \\ b_n} \\
    A &= \crvector{a_{11} & \cdots & a_{1n} \\ \vdots & \ddots & \vdots\\ a_{n1} & \cdots & a_{nn} }
\end{align*}

\subsection*{Transfer Function}
\begin{align*}
    \func{Q}[s] &= \func{Adj}[S \mathbb{I} - A] \\
    \func{r}[s] &= \func{Det}[S \mathbb{I} - A] \\
    \func{G}[s] &= \left[ C {(S \mathbb{I} - A)}^{-1} B + D \right] \\
                &= \frac{C \func{Q}[s] B}{\func{r}[s]} + D \\
                &= \frac{C \func{Q}[s] B + D \func{r}[s]}{\func{r}[s]} \\
    \text{Zeroes} &: C \func{Q}[s] B + D \func{r}[s] = 0 \\
    \text{Poles} &: \func{r}[s] = 0 \quad (\text{Eigenvalues of } A)
\end{align*}

\subsection*{Impulse Response}
\begin{align*}
    \func{h}[t] &= C e^{At} B + D\,\delta(t)
\end{align*}

\subsection*{Matrix-Vector Form}
\begin{align*}
    \func{\underline{x}}[t] &= e^{At}\func{\underline{x}}[0] + \intdef{0}{t}{e^{A(t-\tau)}B\func{u}[\tau]}{\tau} \\
    \func{y}[t] &= C\func{\underline{x}}[t] + D\func{u}[t]
\end{align*}

\section*{Heaviside Step Function}
\subsection*{Transfer Function}
\[
    \laplace{u(t)} = \frac{1}{s}
\]

\subsection*{Impulse Response}
\[
    \deriv{t}{u(t)} = \delta(t)
\]

\subsection*{Matrix-Vector Form}
\[
    \underline{u}[t] =
    \begin{cases}
        \underline{0} & t < 0 \\
        \underline{1} & t \geq 0
    \end{cases}
\]

\section*{Dirac Delta Function}
\subsection*{Transfer Function}
\[
    \laplace{\delta(t)} = 1
\]

\subsection*{Impulse Response}
\[
    \delta(t) \text{ satisfies } \intdef{-\epsilon}{\epsilon}{\delta(t)}{t} = 1, \quad \forall \epsilon > 0
\]

\subsection*{Matrix-Vector Form}
\[
    \underline{\delta}[t] =
    \crvector{
        \delta(t) \\
        \vdots \\
        \delta(t)
    }
\]

\section*{Step Response}
\subsection*{First Order}
\begin{align*}
    \func{y}[t] &= K \left(1 - e^{-t/\tau}\right) \\
    \tau &: \text{time constant} \\
    t_s &\approx 4\tau \quad (\text{for 2\% criterion})
\end{align*}

\subsection*{Second Order}
\subsubsection*{Poles}
\begin{align*}
    \alpha_1^2 < 4 \alpha_0 &\implies \text{complex conjugates} \\
    \alpha_1^2 = 4 \alpha_0 &\implies \text{repeated real} \\
    \alpha_1^2 > 4 \alpha_0 &\implies \text{real, non-repeated}
\end{align*}
Complex Conjugates:
\begin{align*}
    \alpha_1 &= -2 \sigma \\
    \alpha_0 &= \sigma^2 + \omega_d^2 = |p_1|^2
\end{align*}
Repeated Real:
\begin{align*}
    t_s &= \frac{6}{|p_1|}
\end{align*}
Real, Non-Repeated:
\begin{align*}
    |p_2| \gg |p_1| &\implies t_s \approx \frac{4}{|p_1|} \\
    (\text{threshold for above is: } & |p_2|>5|p_1|)\\
    |p_2| \approx |p_1| &\implies t_s \approx \frac{6}{|p_1|} \\
    (\text{threshold for above is: } & 1 \leqslant \frac{|p_2|}{|p_1|} \leqslant 1.1)\\
\end{align*}

\subsubsection*{Damped (Critically Damped, \(\zeta = 1\))}
\begin{align*}
    \func{y}[t] &= 1 - (1+\omega_n t)e^{-\omega_n t}
\end{align*}

\subsubsection*{Under-Damped (\(0 < \zeta < 1\))}
\begin{align*}
    \nu &= \arccos{\zeta} \\
    \func{y}[t] &= \func{G}[0]\left[ 1 - \left( \frac{\omega_n}{\omega_d} e^{\sigma t} \sin{\omega_d t + \nu} \right) \right] \\
\end{align*}

\subsubsection*{Natural (Undamped, \(\zeta = 0\))}
\begin{align*}
    \func{y}[t] &= 1 - \cos{\omega_n t}
\end{align*}

\subsubsection*{LHP Zero}
\begin{enumerate}
    \item A zero in the Left Half Plane does not induce an inverse response.
    \item The step response remains monotonic though modified by the zero dynamics.
\end{enumerate}

\subsubsection*{RHP Zero}
\begin{enumerate}
    \item A Right Half Plane Zero causes an initial inverse (non-minimum phase) response.
    \item The response exhibits an undershoot before eventually rising to steady state.
\end{enumerate}

\subsection*{Performance Metrics}
\begin{align*}
    M_p &: \text{Maximum Overshoot } = \frac{\func{y}_{\max}-\func{y}_{ss}}{\func{y}_{ss}} \times \qty{100}{\percent} \\
    t_r &: \text{Rise Time (\qty{10}{\percent} to \qty{90}{\percent} of final value)} \\
    t_c &: \text{Time steady-state is first crossed} \\
    t_p &: \text{Peak Time (time to first peak)} \\
    t_s &: \text{Settling Time (\qty{2}{\percent} criterion)}
\end{align*}

\subsection*{Overshoot}
\begin{align*}
    M_p &= e^{-\frac{\sigma}{\omega_d} \pi} \times \qty{100}{\percent} \\
        &= e^{\frac{-\zeta \pi}{\sqrt{1 - \zeta^2}}} \times \qty{100}{\percent} \\
    t_c &= \frac{\pi - \nu}{\omega_d} \\
    t_p &= \frac{\pi}{\omega_d} \\
    \func{y_p} &= \func{y_{ss}} \left[1+M_p\right]
\end{align*}

\section*{System Zeroes}
\subsection*{Input Absorption}
\begin{enumerate}
    \item System zeroes can absorb certain input dynamics.
    \item A zero at \( s = z \) may cancel an input pole at \( s = z \).
\end{enumerate}

\subsection*{Transient Suppression}
\begin{enumerate}
    \item Appropriately placed zeroes can mitigate transient peaks.
    \item They are used in controller design to improve system performance.
\end{enumerate}

\subsection*{Pole Cancellation}
\begin{enumerate}
    \item Occurs when a system zero cancels a pole in the transfer function.
    \item Ideal cancellation is sensitive to model uncertainties.
\end{enumerate}

\section*{Frequency Response}
\subsection*{Definition}
\begin{align*}
    \func{G}[j\omega] &= \func{G}[s]\Big|_{s=j\omega}, \quad \omega \in \mathbb{R} \\
    \text{Magnitude} &: | \func{G}[j\omega] | \\
    \text{Phase} &: \angle \func{G}[j\omega]
\end{align*}

\subsection*{Quantification}
\begin{enumerate}
    \item \textbf{Gain Margin}: Factor by which gain can be increased before instability.
    \item \textbf{Phase Margin}: Additional phase lag required to reach instability.
    \item These margins and the overall frequency response are visualized using Bode plots.
\end{enumerate}

\section*{Bode Diagrams}
\subsection*{Decibel Units}
\[
    \text{Magnitude (\unit{\decibel})} = 20\oldlog_{10}\left(\left|\func{G}[j\omega]\right|\right)
\]

\subsection*{Shape}
\subsubsection*{Transfer Function}
\[
    \func{G}[s] = \frac{N(s)}{D(s)}
\]

\subsubsection*{Zeroes}
\begin{enumerate}
    \item Each zero contributes a +20 dB/decade slope beyond its break frequency.
\end{enumerate}

\subsubsection*{Poles}
\begin{enumerate}
    \item Each pole contributes a -20 dB/decade slope beyond its break frequency.
\end{enumerate}

\subsubsection*{Gain}
\begin{enumerate}
    \item A constant gain \( K \) shifts the magnitude plot by \(20\oldlog_{10}(K)\) \unit{\decibel}.
\end{enumerate}

\section*{Bode Magnitude Diagrams}
\subsection*{Shape}
\subsubsection*{Transfer Function}
\[
    \left| \func{G}[j\omega] \right|
\]

\subsubsection*{Zeroes}
\begin{enumerate}
    \item Zeroes add positive slopes to the magnitude plot.
\end{enumerate}

\subsubsection*{Poles}
\begin{enumerate}
    \item Poles add negative slopes to the magnitude plot.
\end{enumerate}

\subsubsection*{Gain}
\begin{enumerate}
    \item The overall gain sets the baseline level of the magnitude plot.
\end{enumerate}

\end{multicols*}
\end{document}