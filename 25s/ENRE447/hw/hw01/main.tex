\documentclass{article}

\usepackage{fancyhdr}
\usepackage{extramarks}
\usepackage{amsmath}
\usepackage{amsthm}
\usepackage{amsfonts}
\usepackage{amssymb}
\usepackage{xparse}
\usepackage{tikz}
\usepackage{graphicx}
\usepackage[plain]{algorithm}
\usepackage{algpseudocode}
\usepackage{listings}
\usepackage{hyperref}
\usepackage[per-mode = fraction]{siunitx}
\usepackage{calc}

\usetikzlibrary{automata,positioning}

\hypersetup{
    colorlinks=true,
    linkcolor=blue,
    filecolor=magenta,
    urlcolor=blue,
    }

\urlstyle{same}

%
% Basic Document Settings
%

\topmargin=-0.45in
\evensidemargin=0in
\oddsidemargin=0in
\textwidth=6.5in
\textheight=9.0in
\headsep=0.25in

\linespread{1.1}

\pagestyle{fancy}
\lhead{\hmwkAuthorName}
\chead{\hmwkClass\ (\hmwkClassInstructor,\ \hmwkClassTime): \hmwkTitle}
\rhead{\firstxmark}
\lfoot{\lastxmark}
\cfoot{\thepage}

\renewcommand\headrulewidth{0.4pt}
\renewcommand\footrulewidth{0.4pt}

\setlength\parindent{0pt}
\allowdisplaybreaks
%
% Title Page
%

\title{
	\vspace{2in}
	\textmd{\textbf{\hmwkClass:\ \hmwkTitle}}\\
	\normalsize\vspace{0.1in}\small{Due\ on\ \hmwkDueDate\ at \hmwkDueTime}\\
	\vspace{0.1in}\large{\textit{\hmwkClassInstructor,\ \hmwkClassTime}}
	\vspace{3in}
}
\author{\textbf{\hmwkAuthorName}}
\date{\hmwkCompletionDate}

%
% Create Problem Sections
%

\newcommand{\enterProblemHeader}[1]{
	\nobreak\extramarks{}{Problem #1 continued on next page\ldots}\nobreak{}
	\nobreak\extramarks{Problem #1 (continued)}{Problem #1 continued on next page\ldots}\nobreak{}
}

\newcommand{\exitProblemHeader}[1]{
	\nobreak\extramarks{Problem #1 (continued)}{Problem #1 continued on next page\ldots}\nobreak{}
	\nobreak\extramarks{Problem #1}{}\nobreak{}
}

%
% Homework Problem Environment
%
\NewDocumentEnvironment{hwkProblem}{m m s}{
	\section*{Problem #1: #2}
	\enterProblemHeader{#1}
	\setcounter{partCounter}{1}
}{
	\exitProblemHeader{#1}
	\IfBooleanF{#3} % if star, no new page
		{\newpage}
}

% Alias for the Solution section header
\newcommand{\hwkSol}{\vspace{\baselineskip / 2}\textbf{\Large Solution}\vspace{\baselineskip / 2}}

% Alias for the Solution Part subsection header
\newcounter{partCounter}
\newcommand{\hwkPart}{
	\vspace{\baselineskip / 2}
	\textbf{\large Part \Alph{partCounter}}
	\vspace{\baselineskip / 2}
	\stepcounter{partCounter}
}

%
% Various Helper Commands
%

% Such That
\newcommand{\st}{\text{s.t.}}

% Useful for algorithms
\newcommand{\alg}[1]{\textsc{\bfseries \footnotesize #1}}

% For derivatives
\newcommand{\deriv}[1]{\frac{\mathrm{d}}{\mathrm{d}x} (#1)}

% For partial derivatives
\newcommand{\pderiv}[2]{\frac{\partial}{\partial #1} (#2)}

% Integral dx
\newcommand{\dx}{\mathrm{d}x}
\newcommand{\dy}{\mathrm{d}y}

% Probability commands: Expectation, Variance, Covariance, Bias
\newcommand{\e}[1]{\mathrm{e}#1}
\newcommand{\E}{\mathrm{E}}
\newcommand{\Var}{\mathrm{Var}}
\newcommand{\Cov}{\mathrm{Cov}}
\newcommand{\Bias}{\mathrm{Bias}}

% Defining Units that are not in the SI base
\DeclareSIUnit\bar{bar}
\DeclareSIUnit\ft{ft}
\DeclareSIUnit\dollar{\$}
\DeclareSIUnit\cent{\text{\textcent}}
\DeclareSIUnit\c{\degreeCelsius}

% Code Listing config
\usepackage{xcolor}
\definecolor{codegreen}{rgb}{0,0.6,0}
\definecolor{codegray}{rgb}{0.5,0.5,0.5}
\definecolor{codepurple}{rgb}{0.58,0,0.82}
\definecolor{backcolour}{rgb}{0.95,0.95,0.92}
\lstdefinestyle{overleaf}{
	% backgroundcolor=\color{backcolour},
	commentstyle=\color{codegreen},
	keywordstyle=\color{magenta},
	numberstyle=\tiny\color{codegray},
	stringstyle=\color{codepurple},
	basicstyle=\ttfamily\footnotesize,
	breakatwhitespace=false,
	breaklines=true,
	captionpos=b,
	keepspaces=true,
	numbers=left,
	numbersep=5pt,
	showspaces=false,
	showstringspaces=false,
	showtabs=false,
	tabsize=4
}

\usepackage[latte]{catppuccinpalette}
\lstdefinestyle{catppuccin}{
	breaklines=true,
	keepspaces=true,
	numbers=left,
	numbersep=5pt,
	showspaces=false,
	showstringspaces=false,
	breakatwhitespace=true,
	tabsize=4,
	stringstyle = {\color{CtpGreen}},
	commentstyle={\color{CtpOverlay1}},
	basicstyle = {\small\color{CtpText}\ttfamily},
	keywordstyle = {\color{CtpMauve}},
	keywordstyle = [2]{\color{CtpBlue}},
	keywordstyle = [3]{\color{CtpYellow}},
	keywordstyle = [4]{\color{CtpLavender}},
	keywordstyle = [5]{\color{CtpPeach}},
	keywordstyle = [6]{\color{CtpTeal}}
}

\lstset{style=catppuccin}


%
% Homework Details
%   - Title
%   - Due date
%   - Due time
%   - Course
%   - Section/Time
%   - Instructor
%   - Author
%

\newcommand{\hmwkTitle}{Homework 01}
\newcommand{\hmwkSubTitle}{}
\newcommand{\hmwkDueDate}{February 10th, 2025}
\newcommand{\hmwkDueTime}{03:30 PM}
\newcommand{\hmwkClass}{ENRE 447 - 0101}
\newcommand{\hmwkClassTime}{03:30 PM}
\newcommand{\hmwkClassInstructor}{Dr. Groth}
\newcommand{\hmwkAuthorName}{\textbf{Vai Srivastava}}
\newcommand{\hmwkCompletionDate}{\today}

\begin{document}

\maketitle

\pagebreak

\begin{hwkProblem}{1}{}

	Select a system and application that you are familiar with. In 2-3 paragraphs, please:
	\begin{enumerate}
		\item Briefly describe the system and define its mission.
		\item List 3 potential benefits/value of formally analyzing reliability.
	\end{enumerate}

	\hwkSol

	\hwkPart

	The system under consideration is my Undergraduate Honors Research Project under Dr. JAY: a magnetic nozzle integrated with a railgun designed to focus the plume of an ionized beam of helium-3 (He-3) at 2 kV. In this configuration, the railgun uses electromagnetic forces to both ionize the He-3 and accelerate it, while the magnetic nozzle is tasked with focusing and directing the high-energy plume. This precise beam control is critical in applications where the directed ion flow is used for propulsion or energy transfer.

	\hwkPart

	\begin{enumerate}
		\item A formal reliability analysis will enable us to quantitatively predict the system’s behavior under a range of operational scenarios. This means we can estimate the likelihood that the magnetic nozzle will perform its function correctly over time, providing data to guide design decisions early in the development phase.
		\item By systematically analyzing reliability, potential failure points (such as loss of magnetic field strength, misalignment of the beam, or material degradation under high-energy conditions) can be identified and addressed. This proactive approach leads to improvements in system design and helps avoid costly or time-intensive redesigns later.
		\item Formally understanding the reliability of the magnetic nozzle contributes to reducing overall risk by anticipating and planning for maintenance needs and potential downtime. This not only enhances safety for any integrated systems (like spacecraft that may be affected by plume impingement) but also helps control costs through better lifecycle management and fewer unexpected repairs.
	\end{enumerate}

\end{hwkProblem}

\begin{hwkProblem}{2}{}

	Discuss the relationship between reliability and failure, both in words and mathematically.

	\hwkSol

	Reliability describes the ability of a system to operate without failure for a desired period or amount of cycles under the desired conditions. Failure goes hand-in-hand with reliability, being the condition we want to avoid: the system no longer being able to operate.

	Mathematically, reliability and failure are related by the following equation:
	\[
		R(t)=1-F(t)
	.\]

\end{hwkProblem}

\begin{hwkProblem}{3}{}

	Let \(A=\{-2,0,1\},\; B=\{0,\pi,\sqrt{2},3\}\), and \(C=\{\pi,\sqrt{2},3\}\). Find
	\begin{enumerate}
		\item \(A \cap C\)
		\item \(\bar{A} \cup B\)
		\item \(\overline{A \cap B}\)
	\end{enumerate}

	\hwkSol

	\hwkPart

	Since none of the elements of \(A\) appear in \(C\),
	\begin{align*}
		A \cap C &= \varnothing \qed
	\end{align*}

	\hwkPart

	With \(U=A\cup B\cup C\), we have 
	\[
		\bar{A} = U\setminus A = \{\pi,\sqrt{2},3\} = C.
	\]
	Thus,
	\begin{align*}
		\bar{A} \cup B &= C \cup B = \{0,\pi,\sqrt{2},3\} \quad (\text{since } C\subset B) \qed
	\end{align*}

	\hwkPart

	First, note that
	\[
		A \cap B = \{0\}.
	\]
	Then,
	\begin{align*}
		\overline{A \cap B} &= U\setminus \{0\} \\
				    &= \{-2,1,\pi,\sqrt{2},3\} \qed
	\end{align*}

\end{hwkProblem}

\begin{hwkProblem}{4}{}

	Simplify the following Boolean expressions.
	\begin{enumerate}
		\item \(\overline{\overline{(A \cap B) \cup C} \cap \bar{B}}\)
		\item \(((A \cup B) \cap \bar{A}) \cup \overline{B \cap A}\)
	\end{enumerate}

	\hwkSol

	\hwkPart

	We start with
	\[
		\overline{\overline{(A \cap B) \cup C} \cap \bar{B}}.
	\]
	Apply De Morgan's law:
	\begin{align*}
		\overline{\overline{(A \cap B) \cup C} \cap \bar{B}} 
		&\simeq \overline{\overline{(A \cap B) \cup C}} \cup \overline{\bar{B}}\\[1mm]
		&\simeq \Bigl((A \cap B) \cup C\Bigr) \cup B\\[1mm]
		&\simeq B \cup C, \qed
	\end{align*}
	since \(B\cup (A\cap B)=B\).

	\hwkPart

	First, simplify the first term:
	\begin{align*}
		(A \cup B) \cap \bar{A} 
		&\simeq (A \cap \bar{A}) \cup (B \cap \bar{A})\\[1mm]
		&\simeq \varnothing \cup (B \cap \bar{A})\\[1mm]
		&\simeq B \cap \bar{A}.
	\end{align*}
	Also, by De Morgan's law,
	\[
		\overline{B \cap A} \simeq \bar{B} \cup \bar{A}.
	\]
	Thus, the overall expression becomes:
	\begin{align*}
		(B \cap \bar{A}) \cup (\bar{B} \cup \bar{A}) 
		&\simeq \bar{A} \cup \bar{B}, \qed
	\end{align*}
	since \(B \cap \bar{A}\subset \bar{A}\)

\end{hwkProblem}

\begin{hwkProblem}{5}{}

	Determine the probability \(p\) for each of the following events:
	\begin{enumerate}
		\item A king, ace, jack of clubs, or queen of diamonds appears in drawing a single card from a well-shuffled ordinary deck.
		\item The sum 8 appears in a single toss of a pair of fair dice.
		\item A non-defective bolt is found, given that out of 600 bolts examined, 12 were defective.
		\item A 7 or 11 comes up in a single toss of a pair of fair dice.
	\end{enumerate}

	\hwkSol

	\hwkPart

	In a 52-card deck:
	\begin{align*}
		\text{Kings: }4,\quad \text{Aces: }4,\quad \text{Jack of Clubs: }1,\quad \text{Queen of Diamonds: }1.
	\end{align*}
	Total favorable outcomes: 
	\[
		4+4+1+1=10 \qed
	\]
	Thus,
	\begin{align*}
		p &= \frac{10}{52} = \frac{5}{26} \qed
	\end{align*}

	\hwkPart

	The outcomes for sum 8 are:
	\[
		(2,6),\,(3,5),\,(4,4),\,(5,3),\,(6,2) \quad (5\,\text{outcomes}).
	\]
	Therefore,
	\begin{align*}
		p &= \frac{5}{36} \qed
	\end{align*}

	\hwkPart

	With 600 bolts and 12 defective, the non-defective bolts number:
	\[
		600-12=588.
	\]
	Thus,
	\begin{align*}
		p &= \frac{588}{600} = 0.98 \qed
	\end{align*}

	\hwkPart

	For a pair of dice, outcomes for 7 are 6 and for 11 are 2. Hence,
	\begin{align*}
		p &= \frac{6+2}{36} = \frac{8}{36} = \frac{2}{9} \qed
	\end{align*}

\end{hwkProblem}

\begin{hwkProblem}{6}{}
	If \(A\) and \(B\) are independent, prove that:
	\begin{enumerate}
		\item \(A\) and \(\bar{B}\) are independent.
		\item \(\bar{A}\) and \(B\) are independent.
		\item \(\bar{A}\) and \(\bar{B}\) are independent.
	\end{enumerate}

	\hwkSol

	By definition, independence means 
	\[
		\Pr(A\cap B)=\Pr(A)\Pr(B).
	\]

	\hwkPart

	\begin{align*}
		\Pr(A) &= \Pr(A\cap B)+\Pr(A\cap \bar{B}) \\
		\Pr(A\cap \bar{B}) &= \Pr(A)-\Pr(A\cap B) \\
				   &\simeq \Pr(A) - \Pr(A)\Pr(B) \quad (\text{substituting } \Pr(A\cap B)=\Pr(A)\Pr(B))\\[1mm]
				   &= \Pr(A)(1-\Pr(B)) \\
				   &= \Pr(A)\Pr(\bar{B}). \qed
	\end{align*}

	\hwkPart

	\begin{align*}
		\Pr(B) &= \Pr(A\cap B)+\Pr(\bar{A}\cap B), \\
		\Pr(\bar{A}\cap B) &= \Pr(B) - \Pr(A\cap B) \\
				   &\simeq \Pr(B) - \Pr(A)\Pr(B) \quad (\text{substituting independence})\\[1mm]
				   &= \Pr(B)(1-\Pr(A)) \\
				   &= \Pr(B)\Pr(\bar{A}). \qed
	\end{align*}

	\hwkPart

	\begin{align*}
		\Pr(\bar{A}\cap \bar{B}) 
		&= 1-\Pr(A\cup B) \\
		&= 1-\Bigl[\Pr(A)+\Pr(B)-\Pr(A\cap B)\Bigr] \\
		&\simeq 1-\Bigl[\Pr(A)+\Pr(B)-\Pr(A)\Pr(B)\Bigr] \quad (\text{substituting independence})\\[1mm]
		&= \bigl(1-\Pr(A)\bigr)\bigl(1-\Pr(B)\bigr) \\
		&= \Pr(\bar{A})\Pr(\bar{B}). \qed
	\end{align*}

\end{hwkProblem}

\begin{hwkProblem}{7}{}

	Use both Equations (2.26: Addition Law of Probability for \(n\) Events) and (2.29: Addition Law of Probability for \(n\) Independent Events) to find 
	\[
		\operatorname{Pr}(s)=\operatorname{Pr}\left(E_1 \cup E_2 \cup E_3\right),
	\]
	for
	\[
		\operatorname{Pr}(E_1)=0.8,\quad \operatorname{Pr}(E_2)=0.9,\quad \operatorname{Pr}(E_3)=0.95.
	\]
	Which equation is preferred for numerical solution?

	\hwkSol

	\hwkPart

	Using Equation (2.26): \emph{(Inclusion-Exclusion)}
	\begin{align*}
		\Pr(E_1\cup E_2\cup E_3) &= \Pr(E_1)+\Pr(E_2)+\Pr(E_3)\\[1mm]
					 &\quad -\Pr(E_1\cap E_2)-\Pr(E_1\cap E_3)-\Pr(E_2\cap E_3)\\[1mm]
					 &\quad +\Pr(E_1\cap E_2\cap E_3)\\[1mm]
					 &\simeq 0.8+0.9+0.95 \\
					 &\quad -\Bigl(0.8\cdot 0.9 \;+\; 0.8\cdot 0.95 \;+\; 0.9\cdot 0.95\Bigr)\\[1mm]
					 &\quad +\Bigl(0.8\cdot 0.9\cdot 0.95\Bigr)\\[1mm]
					 &= 2.65 - \Bigl(0.72\;+\; 0.76\;+\; 0.855\Bigr) + 0.684\\[1mm]
					 &= 2.65 - 2.335 + 0.684\\[1mm]
					 &= 0.999. \qed
	\end{align*}

	\hwkPart

	Using Equation (2.29): \emph{(For Independent Events)}
	\begin{align*}
		\Pr(E_1\cup E_2\cup E_3) &= 1 - (1-\Pr(E_1))(1-\Pr(E_2))(1-\Pr(E_3))\\[1mm]
					 &\simeq 1 - \Bigl[(1-0.8)(1-0.9)(1-0.95)\Bigr]\\[1mm]
					 &\simeq 1 - \Bigl[\cancel{1-0.8}\;0.2\cdot \cancel{1-0.9}\;0.1\cdot \cancel{1-0.95}\;0.05\Bigr]\\[1mm]
					 &= 1 - (0.2\cdot 0.1\cdot 0.05)\\[1mm]
					 &= 1 - 0.001\\[1mm]
					 &= 0.999.
	\end{align*}

	\hwkPart

	Both methods yield \(\operatorname{Pr}(s)=0.999\). However, Equation (2.29) is preferred for numerical solution because it is more direct for independent events and avoids multiple subtractions, which can lead to numerical cancellation errors. \qed

\end{hwkProblem}

\begin{hwkProblem}{8}{}

	An electronic assembly consists of two subsystems, \(A\) and \(B\). Out of 100 preliminary checkout tests:
	\begin{enumerate}
		\item Subsystem \(A\) failed 17 times.
		\item Subsystem \(B\) failed 13 times.
		\item Both \(A\) and \(B\) failed together 7 times.
	\end{enumerate}
	Determine:
	\begin{enumerate}
		\item The probability of \(A\) failing, given that \(B\) has failed.
		\item The probability that \(A\) alone fails.
	\end{enumerate}

	\hwkSol

	Let the probabilities be estimated from frequencies.

	\hwkPart

	The conditional probability is
	\begin{align*}
		\Pr(A\mid B) &= \frac{\Pr(A\cap B)}{\Pr(B)}\\[1mm]
			     &\simeq \frac{7/100}{13/100}\\[1mm]
			     &= \frac{7}{13}\approx 0.5385. \qed
	\end{align*}

	\hwkPart

	The probability of \(A\) failing alone is
	\begin{align*}
		\Pr(A\setminus B) &= \Pr(A)-\Pr(A\cap B)\\[1mm]
				  &= \frac{17}{100} - \frac{7}{100}\\[1mm]
				  &= \frac{10}{100} = 0.1. \qed
	\end{align*}

\end{hwkProblem}

\end{document}
