\documentclass{article}

\usepackage{fancyhdr}
\usepackage{extramarks}
\usepackage{amsmath}
\usepackage{amsthm}
\usepackage{amsfonts}
\usepackage{amssymb}
\usepackage{xparse}
\usepackage{tikz}
\usepackage{graphicx}
\usepackage[plain]{algorithm}
\usepackage{algpseudocode}
\usepackage{listings}
\usepackage{hyperref}
\usepackage[per-mode = fraction]{siunitx}
\usepackage{calc}

\usetikzlibrary{automata,positioning}

\hypersetup{
    colorlinks=true,
    linkcolor=blue,
    filecolor=magenta,
    urlcolor=blue,
    }

\urlstyle{same}

%
% Basic Document Settings
%

\topmargin=-0.45in
\evensidemargin=0in
\oddsidemargin=0in
\textwidth=6.5in
\textheight=9.0in
\headsep=0.25in

\linespread{1.1}

\pagestyle{fancy}
\lhead{\hmwkAuthorName}
\chead{\hmwkClass\ (\hmwkClassInstructor,\ \hmwkClassTime): \hmwkTitle}
\rhead{\firstxmark}
\lfoot{\lastxmark}
\cfoot{\thepage}

\renewcommand\headrulewidth{0.4pt}
\renewcommand\footrulewidth{0.4pt}

\setlength\parindent{0pt}
\allowdisplaybreaks
%
% Title Page
%

\title{
	\vspace{2in}
	\textmd{\textbf{\hmwkClass:\ \hmwkTitle}}\\
	\normalsize\vspace{0.1in}\small{Due\ on\ \hmwkDueDate\ at \hmwkDueTime}\\
	\vspace{0.1in}\large{\textit{\hmwkClassInstructor,\ \hmwkClassTime}}
	\vspace{3in}
}
\author{\textbf{\hmwkAuthorName}}
\date{\hmwkCompletionDate}

%
% Create Problem Sections
%

\newcommand{\enterProblemHeader}[1]{
	\nobreak\extramarks{}{Problem #1 continued on next page\ldots}\nobreak{}
	\nobreak\extramarks{Problem #1 (continued)}{Problem #1 continued on next page\ldots}\nobreak{}
}

\newcommand{\exitProblemHeader}[1]{
	\nobreak\extramarks{Problem #1 (continued)}{Problem #1 continued on next page\ldots}\nobreak{}
	\nobreak\extramarks{Problem #1}{}\nobreak{}
}

%
% Homework Problem Environment
%
\NewDocumentEnvironment{hwkProblem}{m m s}{
	\section*{Problem #1: #2}
	\enterProblemHeader{#1}
	\setcounter{partCounter}{1}
}{
	\exitProblemHeader{#1}
	\IfBooleanF{#3} % if star, no new page
		{\newpage}
}

% Alias for the Solution section header
\newcommand{\hwkSol}{\vspace{\baselineskip / 2}\textbf{\Large Solution}\vspace{\baselineskip / 2}}

% Alias for the Solution Part subsection header
\newcounter{partCounter}
\newcommand{\hwkPart}{
	\vspace{\baselineskip / 2}
	\textbf{\large Part \Alph{partCounter}}
	\vspace{\baselineskip / 2}
	\stepcounter{partCounter}
}

%
% Various Helper Commands
%

% Such That
\newcommand{\st}{\text{s.t.}}

% Useful for algorithms
\newcommand{\alg}[1]{\textsc{\bfseries \footnotesize #1}}

% For derivatives
\newcommand{\deriv}[1]{\frac{\mathrm{d}}{\mathrm{d}x} (#1)}

% For partial derivatives
\newcommand{\pderiv}[2]{\frac{\partial}{\partial #1} (#2)}

% Integral dx
\newcommand{\dx}{\mathrm{d}x}
\newcommand{\dy}{\mathrm{d}y}

% Probability commands: Expectation, Variance, Covariance, Bias
\newcommand{\e}[1]{\mathrm{e}#1}
\newcommand{\E}{\mathrm{E}}
\newcommand{\Var}{\mathrm{Var}}
\newcommand{\Cov}{\mathrm{Cov}}
\newcommand{\Bias}{\mathrm{Bias}}

% Defining Units that are not in the SI base
\DeclareSIUnit\bar{bar}
\DeclareSIUnit\ft{ft}
\DeclareSIUnit\dollar{\$}
\DeclareSIUnit\cent{\text{\textcent}}
\DeclareSIUnit\c{\degreeCelsius}

% Code Listing config
\usepackage{xcolor}
\definecolor{codegreen}{rgb}{0,0.6,0}
\definecolor{codegray}{rgb}{0.5,0.5,0.5}
\definecolor{codepurple}{rgb}{0.58,0,0.82}
\definecolor{backcolour}{rgb}{0.95,0.95,0.92}
\lstdefinestyle{overleaf}{
	% backgroundcolor=\color{backcolour},
	commentstyle=\color{codegreen},
	keywordstyle=\color{magenta},
	numberstyle=\tiny\color{codegray},
	stringstyle=\color{codepurple},
	basicstyle=\ttfamily\footnotesize,
	breakatwhitespace=false,
	breaklines=true,
	captionpos=b,
	keepspaces=true,
	numbers=left,
	numbersep=5pt,
	showspaces=false,
	showstringspaces=false,
	showtabs=false,
	tabsize=4
}

\usepackage[latte]{catppuccinpalette}
\lstdefinestyle{catppuccin}{
	breaklines=true,
	keepspaces=true,
	numbers=left,
	numbersep=5pt,
	showspaces=false,
	showstringspaces=false,
	breakatwhitespace=true,
	tabsize=4,
	stringstyle = {\color{CtpGreen}},
	commentstyle={\color{CtpOverlay1}},
	basicstyle = {\small\color{CtpText}\ttfamily},
	keywordstyle = {\color{CtpMauve}},
	keywordstyle = [2]{\color{CtpBlue}},
	keywordstyle = [3]{\color{CtpYellow}},
	keywordstyle = [4]{\color{CtpLavender}},
	keywordstyle = [5]{\color{CtpPeach}},
	keywordstyle = [6]{\color{CtpTeal}}
}

\lstset{style=catppuccin}


%
% Homework Details
%   - Title
%   - Subtitle
%   - Due date
%   - Due time
%   - Course
%   - Section/Time
%   - Instructor
%   - Author
%

\newcommand{\hmwkTitle}{HW 04}
\newcommand{\hmwkSubTitle}{Assignment 4}
\newcommand{\hmwkDueDate}{February 27th, 2025}
\newcommand{\hmwkDueTime}{11:59 PM}
\newcommand{\hmwkClass}{PHYS 313}
\newcommand{\hmwkClassTime}{0101}
\newcommand{\hmwkClassInstructor}{Dr. Ji}
\newcommand{\hmwkAuthorName}{\textbf{Vai Srivastava}}
\newcommand{\hmwkCompletionDate}{\today}

\begin{document}

\maketitle

\pagebreak

\begin{hwkProblem}{2.22}{}

	Find the potential a distance \( s \) from an infinitely long straight wire that carries a uniform line charge \( \lambda \). Compute the gradient of your potential, and check that it yields the correct field.

	\hwkSol

	\begin{align*}
		V(s)-V(s_0) &= -\int_{s_0}^{s} \frac{2k\lambda}{s'}\,ds' \\
			    &= -2k\lambda \ln\frac{s}{s_0} \\
		V(s) &= 2k\lambda \ln\frac{s_0}{s} \quad \text{(choosing } V(s_0)=0\text{)} \\
		\frac{dV}{ds} &= -\frac{2k\lambda}{s} \quad\Longrightarrow\quad \nabla V = -\frac{2k\lambda}{s}\,\hat{s} \\
		-\nabla V &= \frac{2k\lambda}{s}\,\hat{s} = \mathbf{E}(s)
	\end{align*}

	\[
		\boxed{
			V(\rcurs(s)) = 2k\lambda \ln\frac{s_0}{\rcurs(s)} \quad,\quad
			\mathbf{E}(\rcurs(s)) = \frac{2k\lambda}{\rcurs(s)}\,\brcurs(s)
		}
	\]

\end{hwkProblem}
\begin{hwkProblem}{2.26}{}

	A conical surface (an empty ice-cream cone) carries a uniform surface charge \( \sigma \). The height of the cone is \( h \), as is the radius of the top. Find the potential difference between points \( \bm{a} \) (the vertex) and \( \bm{b} \) (the center of the top).

	\hwkSol

	\begin{align*}
		\text{For point } \bm{a}\text{ (vertex):}\quad
		V(\bm{a}) &= k\sigma \int_0^{2\pi} d\theta \int_0^{L} \frac{dA}{r_a}
		\quad,\quad r_a = s,\; dA = s\sin\alpha\, ds\, d\theta\\[1mm]
			  &= k\sigma \int_0^{2\pi} d\theta \int_0^{L} \frac{s\sin\alpha\, ds}{s}
			  = 2\pi k\sigma \sin\alpha \,L\\[1mm]
			  \intertext{For a cone with height \(h\) and top radius \(h\):}
		\tan\alpha &= \frac{h}{h}=1 \quad\Longrightarrow\quad \alpha=45^\circ,\quad \sin\alpha=\frac{1}{\sqrt{2}},\quad L = \sqrt{h^2+h^2}=h\sqrt{2}\\[1mm]
		\therefore\quad V(\bm{a}) &= 2\pi k\sigma \frac{1}{\sqrt{2}}\, (h\sqrt{2})
		= 2\pi k\sigma h\\[3mm]
		\text{For point } \bm{b}\text{ (center of top):}\quad
		V(\bm{b}) &= k\sigma \int_0^{2\pi} d\theta \int_0^{L} \frac{dA}{r_b(s)}\\[1mm]
		\text{where } r_b(s) &= \sqrt{s^2 - \sqrt{2}\,h\, s + h^2}
		=\sqrt{\Bigl(s-\frac{h}{\sqrt{2}}\Bigr)^2+\frac{h^2}{2}}\\[1mm]
		\text{Let } u &= s-\frac{h}{\sqrt{2}},\quad ds=du,\quad s = u+\frac{h}{\sqrt{2}},\\[1mm]
		\text{with } u: -\frac{h}{\sqrt{2}} \to \frac{h}{\sqrt{2}},\quad dA = \frac{s}{\sqrt{2}}\,du\,d\theta\\[1mm]
		\implies\; V(\bm{b}) &= \frac{2\pi k\sigma}{\sqrt{2}}
		\int_{-h/\sqrt{2}}^{h/\sqrt{2}} \frac{u+\frac{h}{\sqrt{2}}}{\sqrt{u^2+\frac{h^2}{2}}}\,du\\[1mm]
				     &=\frac{2\pi k\sigma}{\sqrt{2}}\left[
					     \underbrace{\int_{-h/\sqrt{2}}^{h/\sqrt{2}} \frac{u\,du}{\sqrt{u^2+\frac{h^2}{2}}}}_{=0}
					     +\frac{h}{\sqrt{2}} \int_{-h/\sqrt{2}}^{h/\sqrt{2}} \frac{du}{\sqrt{u^2+\frac{h^2}{2}}}
				     \right]\\[1mm]
				     &=\frac{2\pi k\sigma h}{2}
				     \cdot 2\int_{0}^{h/\sqrt{2}} \frac{du}{\sqrt{u^2+\frac{h^2}{2}}}\\[1mm]
				     &=\pi k\sigma h \cdot 2\ln\left(\frac{h/\sqrt{2}+\sqrt{(h/\sqrt{2})^2+\frac{h^2}{2}}}{h/\sqrt{2}}\right)\\[1mm]
				     &=\; 2\pi k\sigma h \ln\left(\frac{h/\sqrt{2}+h}{h/\sqrt{2}}\right)
				     = 2\pi k\sigma h \ln\left(\sqrt{2}+1\right)
	\end{align*}

	\[
		\boxed{V(\brcurs{b})-V(\brcurs{a}) = 2\pi k\sigma h \Bigl[\ln\left(\sqrt{2}+1\right)-1\Bigr]}
	\]

\end{hwkProblem}
\begin{hwkProblem}{2.28}{}

	Use the following equation to calculate the potential inside a uniformly charged solid sphere of radius \( R \) and total charge \( q \):
	\[
		\func{V}[r] = \frac{1}{4\pi\epsilon_{0}}\int\frac{\func{\rho}[\bm{r'}]}{\rcurs}\text{d}\tau'
	\]

	\hwkSol

	\begin{align*}
		\rho &= \frac{q}{\frac{4}{3}\pi R^3} = \frac{3q}{4\pi R^3} \\[2mm]
		V(r) &= \frac{1}{4\pi\epsilon_0}\left[
			\frac{1}{r}\int_{0}^{r}\rho\,(4\pi r'^2\,dr')
			+\int_{r}^{R}\frac{\rho\,(4\pi r'^2\,dr')}{r'}
		\right] \\[2mm]
		     &= \frac{1}{4\pi\epsilon_0}\left[
			     \frac{4\pi\rho}{r}\left(\frac{r^3}{3}\right)
			     +4\pi\rho\int_{r}^{R}r'\,dr'
		     \right] \\[2mm]
		     &= \frac{1}{4\pi\epsilon_0}\left[
			     \frac{4\pi\rho\,r^2}{3}
			     +4\pi\rho\left(\frac{R^2-r^2}{2}\right)
		     \right] \\[2mm]
		     &= \frac{\rho}{\epsilon_0}\left(\frac{r^2}{3}+\frac{R^2-r^2}{2}\right) \\[2mm]
		     &= \frac{3q}{4\pi R^3\epsilon_0}\left(\frac{r^2}{3}+\frac{R^2-r^2}{2}\right) \\[2mm]
		     &= \frac{q}{4\pi\epsilon_0R}\left(\frac{3}{2}-\frac{r^2}{2R^2}\right)
	\end{align*}

	\[
		\boxed{V(\rcurs(r)) = \frac{q}{4\pi\epsilon_0R}\left(\frac{3}{2}-\frac{r^2}{2R^2}\right)}
	\]

\end{hwkProblem}
\begin{hwkProblem}{2.33}{}

	Consider an infinite chain of point charges, \( \pm q \) (with alternating signs), strung out along the \( x \) axis, each a distance \( a \) from its nearest neighbors. Find the work per particle required to assemble this system.

	\hwkSol

	\begin{align*}
		\text{For a charge at } x=0:\quad V(0) &= \frac{1}{4\pi\epsilon_0}\sum_{n\neq0}\frac{q_n}{|na|} \\
						       &= \frac{1}{4\pi\epsilon_0}\Biggl[\frac{-q}{a}+\frac{-q}{a}+\frac{q}{2a}+\frac{q}{2a}+\frac{-q}{3a}+\frac{-q}{3a}+\cdots\Biggr] \\
						       &= \frac{q}{4\pi\epsilon_0a}\,2\sum_{n=1}^{\infty}\frac{(-1)^n}{n}
						       = \frac{q}{4\pi\epsilon_0a}\,2\bigl(-\ln2\bigr) \\
						       &= -\frac{q\ln2}{2\pi\epsilon_0a} \\[2mm]
		\text{Work per particle:}\quad U &= \frac{1}{2}\,q\,V(0)
		= -\frac{q^2\ln2}{4\pi\epsilon_0a}
	\end{align*}

	\[
		\boxed{U = -\frac{q^2\ln2}{4\pi\epsilon_0a}}
	\]

\end{hwkProblem}
\begin{hwkProblem}{2.35}{}

	Here is a fourth way of computing the energy of a uniformly charged solid sphereL Asseble it like a snowball, layer by layer, each time bringing in an infinitesimal charge \( \text{d}q \) from far away and smearing it uniformly over the surface, thereby increasing the radius. How much work \( \text{d}W \) does it take to build up the radius by an amount \( \text{d}r \)? Integrate this to find the work necessary to create the entire sphere of radius \( R \) and total charge \( q \).

	\hwkSol

	\begin{align*}
		q(r) &= \frac{q}{R^3}r^3 \\
		\text{d}q &= \frac{d}{dr}\left(\frac{q}{R^3}r^3\right)dr = 3q\,\frac{r^2}{R^3}\,dr \\
		V(r) &= \frac{1}{4\pi\epsilon_0}\frac{q(r)}{r} = \frac{1}{4\pi\epsilon_0}\frac{q\,r^2}{R^3} \\
		\text{d}W &= V(r)\,\text{d}q = \frac{1}{4\pi\epsilon_0}\frac{q\,r^2}{R^3}\cdot 3q\,\frac{r^2}{R^3}\,dr = \frac{3q^2}{4\pi\epsilon_0 R^6}r^4\,dr \\
		W &= \int_0^R \text{d}W = \frac{3q^2}{4\pi\epsilon_0 R^6}\int_0^R r^4\,dr = \frac{3q^2}{4\pi\epsilon_0 R^6}\cdot\frac{R^5}{5} \\
		  &= \frac{3q^2}{20\pi\epsilon_0 R}
	\end{align*}

	\[
		\boxed{U = \frac{3q^2}{20\pi\epsilon_0 R}}
	\]

\end{hwkProblem}
\begin{hwkProblem}{2.38}{}

	A metal sphere of radius \( R \), carrying charge \( q \), is surrounded by a thick concentric metal shell (inner radius \( a \), outer radius \( b \)). The shell carries no net charge.
	\begin{enumerate}
		\item Find the surface charge density \( \sigma \) at \( R \), at \( a \), and at \( b \).
		\item Find the potential at the center, using infinity as a reference point.
		\item Now the outer surface is touched to a grounding wire, which drains off charge and lowers its potential to zero (same as at infinity). How do your answers to the previous two parts change?
	\end{enumerate}

	\hwkSol

	\hwkPart

	\begin{align*}
		\sigma_R &= \frac{q}{4\pi R^2},\\[1mm]
		\sigma_a &= -\frac{q}{4\pi a^2},\\[1mm]
		\sigma_b &= \frac{q}{4\pi b^2} \quad \text{(since the shell is neutral: } \, -q + q = 0\text{)}.
	\end{align*}

	\[
		\boxed{
			\sigma_R=\dfrac{q}{4\pi R^2},\quad \sigma_a=-\dfrac{q}{4\pi a^2},\quad \sigma_b=\dfrac{q}{4\pi b^2}
		}
	\]

	\hwkPart

	\begin{align*}
		V(0) &= \frac{1}{4\pi\epsilon_0}\Biggl[
			\frac{q}{R} + \frac{(-q)}{a} + \frac{q}{b}
		\Biggr] \\
		     &= \frac{q}{4\pi\epsilon_0}\left(\frac{1}{R} - \frac{1}{a} + \frac{1}{b}\right).
		\end{align*}

	\[
		\boxed{
			V(0)=\dfrac{q}{4\pi\epsilon_0}\left(\dfrac{1}{R}-\dfrac{1}{a}+\dfrac{1}{b}\right)
		}
	\]

	\hwkPart

	\begin{align*}
		\text{Grounding the outer surface forces } V(b)=0 &\Longrightarrow \text{ its net charge is } Q_b=0.\\[1mm]
		\text{Then, by Gauss' law, the inner surface still has } &-q \quad \text{and the sphere } +q.\\[1mm]
		\text{Thus, the revised surface charge densities are:}\\[1mm]
		\sigma_R &= \frac{q}{4\pi R^2},\\[1mm]
		\sigma_a &= -\frac{q}{4\pi a^2},\\[1mm]
		\sigma_b &= 0.\\[2mm]
		\text{And the potential at the center becomes:}\\[1mm]
		V(0) &= \frac{1}{4\pi\epsilon_0}\left(\frac{q}{R} - \frac{q}{a}\right).
	\end{align*}

	\[
		\boxed{
			\sigma_R=\dfrac{q}{4\pi R^2},\quad \sigma_a=-\dfrac{q}{4\pi a^2},\quad \sigma_b=0,\quad V(0)=\dfrac{q}{4\pi\epsilon_0}\left(\dfrac{1}{R}-\dfrac{1}{a}\right)
		}
	\]

\end{hwkProblem}
\end{document}
