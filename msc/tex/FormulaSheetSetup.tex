\usepackage[utf8]{inputenc}
\usepackage{amsmath}
\usepackage{amssymb}
\usepackage{multicol}
\usepackage{blindtext}
\usepackage{geometry}
\geometry{a4paper, left=5mm, right=5mm, top=5mm, bottom=12mm}
\setlength{\parindent}{0pt}
\setlength{\columnsep}{0.4cm}
\setlength{\columnseprule}{0.5pt}
\usepackage{enumitem}
\usepackage{svg}
\usepackage{dsfont}
\usepackage{hyperref}

\usepackage{fancyhdr}
\usepackage{extramarks}
\usepackage{amsthm}
\usepackage{amsfonts}
\usepackage{xparse}
\usepackage{bm}
\usepackage{tikz}
\usepackage{graphicx}
\usepackage{caption}
\usepackage{subcaption}
\usepackage[plain]{algorithm}
\usepackage{algpseudocode}
\usepackage{listings}
\usepackage[per-mode = fraction]{siunitx}
\usepackage{calc}
\usepackage{cancel}
\usepackage{calligra}

\usepackage{xcolor}
\definecolor{myblue}{cmyk}{1,.72,0,.38}
\definecolor{mygreen}{RGB}{78, 153, 67}     % A vibrant green
\definecolor{myred}{RGB}{220, 53, 69}       % A bright red
\definecolor{myyellow}{cmyk}{0,0,1,0}       % Pure yellow
\definecolor{mydarkblue}{cmyk}{1,.6,0,.4}   % A darker shade of blue
\definecolor{mypurple}{RGB}{104, 33, 122}   % A deep purple
\definecolor{myorange}{RGB}{253, 126, 20}   % A bright orange
\definecolor{mygray}{gray}{0.6}             % A medium gray, 0 is black, 1 is white
\definecolor{myteal}{RGB}{0,128,128}        % A teal color

% TikZ settings for diagrams
\usepackage{tikz}
\usetikzlibrary{shapes,positioning,arrows,fit,calc,graphs,graphs.standard}

% Font and typography settings
\usepackage[T1]{fontenc}
\usepackage[nosf]{kpfonts} % Keeping kpfonts for mathematical symbols
\usepackage{sourcesanspro} % Sans-serif font for clarity

% Microtype for improved typography
\usepackage{microtype}

% Apply color to all math environments
% \everymath{\color{myblue}}
% \everydisplay{\color{myblue}}

% Customized spacing
\renewcommand{\baselinestretch}{.8}

% Page footers
\pagestyle{fancy}
\lfoot{\fsheetCourse: \fsheetExam}
\rfoot{\fsheetAuthor}

% Layer declaration for TikZ drawings
\pgfdeclarelayer{background}
\pgfsetlayers{background,main}

\usepackage{titlesec} % For customizing section titles
% Redefining section titles for a more compact look
\titleformat{\section}
  {\normalfont\large\sffamily\bfseries\color{myblue}}
  {\thesection}{1em}{}
\titleformat{\subsection}
  {\normalfont\normalsize\sffamily\bfseries\color{mygreen}}
  {\thesubsection}{1em}{}
\titleformat{\subsubsection}
  {\normalfont\normalsize\sffamily\bfseries\color{mypurple}}
  {\thesubsubsection}{1em}{}

% Adjust spacing before and after section titles
\titlespacing*{\section}{0pt}{1ex plus 1ex minus .2ex}{0.5ex plus .2ex}
\titlespacing*{\subsection}{0pt}{1ex plus 1ex minus .2ex}{0.5ex plus .2ex}
\titlespacing*{\subsubsection}{0pt}{1ex plus 1ex minus .2ex}{0.5ex plus .2ex}

% Adjust spacing for enumerate environment to make it more compact
\setlist[enumerate]{
    topsep=0pt, % Space between the first item and the preceding paragraph
    partopsep=0pt, % Extra space added to topsep when environment starts a new paragraph
    parsep=0pt, % Space between paragraphs within an item
    itemsep=1pt, % Space between items
    leftmargin=* % Adjusts the left margin to align with the surrounding text
}
\setlist[itemize]{
    topsep=0pt, % Space between the first item and the preceding paragraph
    partopsep=0pt, % Extra space added to topsep when environment starts a new paragraph
    parsep=0pt, % Space between paragraphs within an item
    itemsep=1pt, % Space between items
    leftmargin=* % Adjusts the left margin to align with the surrounding text
}
% Reduce spacing around all display math, affecting align environments
\AtBeginDocument{
  \setlength{\abovedisplayskip}{0pt} % Space above display math
  \setlength{\abovedisplayshortskip}{0pt} % Space above display math when preceding text line is short
  \setlength{\belowdisplayskip}{0pt} % Space below display math
  \setlength{\belowdisplayshortskip}{0pt} % Space below display math when following text line is short
}
