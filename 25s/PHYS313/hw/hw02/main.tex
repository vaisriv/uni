\documentclass{article}

\usepackage{fancyhdr}
\usepackage{extramarks}
\usepackage{amsmath}
\usepackage{amsthm}
\usepackage{amsfonts}
\usepackage{amssymb}
\usepackage{xparse}
\usepackage{tikz}
\usepackage{graphicx}
\usepackage[plain]{algorithm}
\usepackage{algpseudocode}
\usepackage{listings}
\usepackage{hyperref}
\usepackage[per-mode = fraction]{siunitx}
\usepackage{calc}

\usetikzlibrary{automata,positioning}

\hypersetup{
    colorlinks=true,
    linkcolor=blue,
    filecolor=magenta,
    urlcolor=blue,
    }

\urlstyle{same}

%
% Basic Document Settings
%

\topmargin=-0.45in
\evensidemargin=0in
\oddsidemargin=0in
\textwidth=6.5in
\textheight=9.0in
\headsep=0.25in

\linespread{1.1}

\pagestyle{fancy}
\lhead{\hmwkAuthorName}
\chead{\hmwkClass\ (\hmwkClassInstructor,\ \hmwkClassTime): \hmwkTitle}
\rhead{\firstxmark}
\lfoot{\lastxmark}
\cfoot{\thepage}

\renewcommand\headrulewidth{0.4pt}
\renewcommand\footrulewidth{0.4pt}

\setlength\parindent{0pt}
\allowdisplaybreaks
%
% Title Page
%

\title{
	\vspace{2in}
	\textmd{\textbf{\hmwkClass:\ \hmwkTitle}}\\
	\normalsize\vspace{0.1in}\small{Due\ on\ \hmwkDueDate\ at \hmwkDueTime}\\
	\vspace{0.1in}\large{\textit{\hmwkClassInstructor,\ \hmwkClassTime}}
	\vspace{3in}
}
\author{\textbf{\hmwkAuthorName}}
\date{\hmwkCompletionDate}

%
% Create Problem Sections
%

\newcommand{\enterProblemHeader}[1]{
	\nobreak\extramarks{}{Problem #1 continued on next page\ldots}\nobreak{}
	\nobreak\extramarks{Problem #1 (continued)}{Problem #1 continued on next page\ldots}\nobreak{}
}

\newcommand{\exitProblemHeader}[1]{
	\nobreak\extramarks{Problem #1 (continued)}{Problem #1 continued on next page\ldots}\nobreak{}
	\nobreak\extramarks{Problem #1}{}\nobreak{}
}

%
% Homework Problem Environment
%
\NewDocumentEnvironment{hwkProblem}{m m s}{
	\section*{Problem #1: #2}
	\enterProblemHeader{#1}
	\setcounter{partCounter}{1}
}{
	\exitProblemHeader{#1}
	\IfBooleanF{#3} % if star, no new page
		{\newpage}
}

% Alias for the Solution section header
\newcommand{\hwkSol}{\vspace{\baselineskip / 2}\textbf{\Large Solution}\vspace{\baselineskip / 2}}

% Alias for the Solution Part subsection header
\newcounter{partCounter}
\newcommand{\hwkPart}{
	\vspace{\baselineskip / 2}
	\textbf{\large Part \Alph{partCounter}}
	\vspace{\baselineskip / 2}
	\stepcounter{partCounter}
}

%
% Various Helper Commands
%

% Such That
\newcommand{\st}{\text{s.t.}}

% Useful for algorithms
\newcommand{\alg}[1]{\textsc{\bfseries \footnotesize #1}}

% For derivatives
\newcommand{\deriv}[1]{\frac{\mathrm{d}}{\mathrm{d}x} (#1)}

% For partial derivatives
\newcommand{\pderiv}[2]{\frac{\partial}{\partial #1} (#2)}

% Integral dx
\newcommand{\dx}{\mathrm{d}x}
\newcommand{\dy}{\mathrm{d}y}

% Probability commands: Expectation, Variance, Covariance, Bias
\newcommand{\e}[1]{\mathrm{e}#1}
\newcommand{\E}{\mathrm{E}}
\newcommand{\Var}{\mathrm{Var}}
\newcommand{\Cov}{\mathrm{Cov}}
\newcommand{\Bias}{\mathrm{Bias}}

% Defining Units that are not in the SI base
\DeclareSIUnit\bar{bar}
\DeclareSIUnit\ft{ft}
\DeclareSIUnit\dollar{\$}
\DeclareSIUnit\cent{\text{\textcent}}
\DeclareSIUnit\c{\degreeCelsius}

% Code Listing config
\usepackage{xcolor}
\definecolor{codegreen}{rgb}{0,0.6,0}
\definecolor{codegray}{rgb}{0.5,0.5,0.5}
\definecolor{codepurple}{rgb}{0.58,0,0.82}
\definecolor{backcolour}{rgb}{0.95,0.95,0.92}
\lstdefinestyle{overleaf}{
	% backgroundcolor=\color{backcolour},
	commentstyle=\color{codegreen},
	keywordstyle=\color{magenta},
	numberstyle=\tiny\color{codegray},
	stringstyle=\color{codepurple},
	basicstyle=\ttfamily\footnotesize,
	breakatwhitespace=false,
	breaklines=true,
	captionpos=b,
	keepspaces=true,
	numbers=left,
	numbersep=5pt,
	showspaces=false,
	showstringspaces=false,
	showtabs=false,
	tabsize=4
}

\usepackage[latte]{catppuccinpalette}
\lstdefinestyle{catppuccin}{
	breaklines=true,
	keepspaces=true,
	numbers=left,
	numbersep=5pt,
	showspaces=false,
	showstringspaces=false,
	breakatwhitespace=true,
	tabsize=4,
	stringstyle = {\color{CtpGreen}},
	commentstyle={\color{CtpOverlay1}},
	basicstyle = {\small\color{CtpText}\ttfamily},
	keywordstyle = {\color{CtpMauve}},
	keywordstyle = [2]{\color{CtpBlue}},
	keywordstyle = [3]{\color{CtpYellow}},
	keywordstyle = [4]{\color{CtpLavender}},
	keywordstyle = [5]{\color{CtpPeach}},
	keywordstyle = [6]{\color{CtpTeal}}
}

\lstset{style=catppuccin}


%
% Homework Details
%   - Title
%   - Subtitle
%   - Due date
%   - Due time
%   - Course
%   - Section/Time
%   - Instructor
%   - Author
%

\newcommand{\hmwkTitle}{HW 02}
\newcommand{\hmwkSubTitle}{Assignment 2}
\newcommand{\hmwkDueDate}{February 13th, 2025}
\newcommand{\hmwkDueTime}{11:59 PM}
\newcommand{\hmwkClass}{PHYS 313}
\newcommand{\hmwkClassTime}{0101}
\newcommand{\hmwkClassInstructor}{Dr. Ji}
\newcommand{\hmwkAuthorName}{\textbf{Vai Srivastava}}
\newcommand{\hmwkCompletionDate}{\today}

\begin{document}

\maketitle

\pagebreak

\begin{hwkProblem}{1.39}{}

	\begin{enumerate}
		\item Check the divergence theorem for the function \(  \vec{v}_1=r^2\mathbf{\hat{r}} \), using as your volume the sphere of radius \( R \), centered at the origin.
		\item Do the same for \( \vec{v}_2 = \frac{1}{r^2} \mathbf{\hat{r}} \).
	\end{enumerate}

	\hwkSol

	\hwkPart
	\[
		\vec{v}_1 = r^2 \, \hat{r}
	\]
	\[
		\nabla \cdot \vec{v}_1 = \frac{1}{r^2}\frac{\partial}{\partial r}\Bigl(r^2\,(r^2)\Bigr)
		=\frac{1}{r^2}\frac{\partial}{\partial r}(r^4)
		=\frac{4r^3}{r^2}=4r.
	\]
	\[
		\begin{aligned}
			\int_V (\nabla\cdot\vec{v}_1) d\tau &= 4\int_0^R r\, (r^2\sin\theta\,dr\,d\theta\,d\phi)\\[1mm]
							    &= 4\int_0^R r^3\,dr \int_0^\pi \sin\theta\,d\theta \int_0^{2\pi}d\phi\\[1mm]
							    &= 4\left(\frac{R^4}{4}\right)(2)(2\pi)=4\pi R^4.
		\end{aligned}
	\]
	\[
		\text{On the surface: } \vec{v}_1\cdot\hat{n}=R^2,\quad dA=R^2\sin\theta\,d\theta\,d\phi.
	\]
	\[
		\Phi = \int_S R^2\,(R^2\sin\theta\,d\theta\,d\phi)
		=R^4\int_0^\pi\sin\theta\,d\theta\int_0^{2\pi}d\phi
		=4\pi R^4.
	\]
	\hwkPart
	\[
		\vec{v}_2 = \frac{1}{r^2}\,\hat{r}
	\]
	\[
		\nabla\cdot\vec{v}_2 = \frac{1}{r^2}\frac{\partial}{\partial r}\Bigl(r^2\frac{1}{r^2}\Bigr)
		=\frac{1}{r^2}\frac{\partial}{\partial r}(1)=0.
	\]
	\[
		\int_V (\nabla\cdot\vec{v}_2)d\tau=0.
	\]
	\[
		\text{However, on the surface: } v_r=\frac{1}{R^2},\quad \Phi = \frac{1}{R^2}(4\pi R^2)=4\pi.
	\]
	\[
		\text{The discrepancy is due to the singularity at } r=0.
	\]
\end{hwkProblem}

\begin{hwkProblem}{1.43}{}

	\begin{enumerate}
		\item Find the divergence of the function \[ \vec{v} = s \left( 2 + \sin^2{\phi} \right) \mathbf{\hat{s}} + s \sin{\phi} \cos{\phi} \boldsymbol{\hat{\phi}} + 3z \mathbf{\hat{z}} .\]
		\item Test the divergence theorem for this function, using a quarter-cylinder with radius \( r=2, h=5 \).
		\item Find the curl of \( \vec{v} \).
	\end{enumerate}

	\hwkSol

	\hwkPart
	\[
		\vec{v} = s \left( 2 + \sin^2{\phi} \right) \mathbf{\hat{s}} + s \sin{\phi} \cos{\phi} \boldsymbol{\hat{\phi}} + 3z \mathbf{\hat{z}}.
	\]
	\[
		\nabla\cdot\vec{v}=\frac{1}{s}\frac{\partial}{\partial s}\Bigl(s\,v_s\Bigr)
		+\frac{1}{s}\frac{\partial v_\phi}{\partial\phi}+\frac{\partial v_z}{\partial z}.
	\]
	\[
		s\,v_s=s^2(2+\sin^2\phi),\quad \frac{\partial}{\partial s}\Bigl(s^2(2+\sin^2\phi)\Bigr)=2s(2+\sin^2\phi).
	\]
	\[
		\frac{1}{s}\frac{\partial}{\partial s}\Bigl(s\,v_s\Bigr)=2(2+\sin^2\phi).
	\]
	\[
		\frac{\partial v_\phi}{\partial\phi} = s(\cos^2\phi-\sin^2\phi)=s\cos2\phi,\quad \frac{1}{s}\frac{\partial v_\phi}{\partial\phi}=\cos2\phi.
	\]
	\[
		\frac{\partial v_z}{\partial z}=3.
	\]
	\[
		\nabla\cdot\vec{v}=2(2+\sin^2\phi)+\cos2\phi+3.
	\]
	\[
		\text{Using } \cos2\phi=1-2\sin^2\phi,\quad 2(2+\sin^2\phi)=4+2\sin^2\phi.
	\]
	\[
		\nabla\cdot\vec{v}=4+2\sin^2\phi+1-2\sin^2\phi+3=8.
	\]
	\hwkPart
	\[
		\text{Volume of quarter-cylinder: } V=\frac{1}{4}\pi(2)^2(5)=5\pi.
	\]
	\[
		\int_V(\nabla\cdot\vec{v})d\tau=8(5\pi)=40\pi.
	\]
	\[
		\text{Thus, the net flux over the surface is } 40\pi.
	\]
	\hwkPart
	\[
		\nabla\times\vec{v}=
		\begin{pmatrix}
			\hat{s} & \hat{\phi} & \hat{z} \\
			\frac{\partial}{\partial s} & \frac{1}{s}\frac{\partial}{\partial\phi} & \frac{\partial}{\partial z} \\
			v_s & v_\phi & v_z
		\end{pmatrix}.
	\]
	\[
		(\nabla\times\vec{v})_z=\frac{1}{s}\left[\frac{\partial}{\partial s}(s\,v_\phi)-\frac{\partial v_s}{\partial\phi}\right].
	\]
	\[
		s\,v_\phi=s^2\sin\phi\cos\phi,\quad \frac{\partial}{\partial s}(s^2\sin\phi\cos\phi)=2s\sin\phi\cos\phi.
	\]
	\[
		\frac{\partial v_s}{\partial\phi}=\frac{\partial}{\partial\phi}\Bigl(s(2+\sin^2\phi)\Bigr)=2s\sin\phi\cos\phi.
	\]
	\[
		(\nabla\times\vec{v})_z=\frac{1}{s}(2s\sin\phi\cos\phi-2s\sin\phi\cos\phi)=0.
	\]
	\[
		(\nabla\times\vec{v})_s=\frac{1}{s}\frac{\partial v_z}{\partial\phi}-\frac{\partial v_\phi}{\partial z}=0,\quad (\nabla\times\vec{v})_\phi=\frac{\partial v_s}{\partial z}-\frac{\partial v_z}{\partial s}=0.
	\]
	\[
		\therefore \nabla\times\vec{v}=\vec{0}.
	\]
\end{hwkProblem}

\begin{hwkProblem}{1.47}{}

	\begin{enumerate}
		\item Write an expression for the volume charge density of \( \rho \left( \vec{r} \right) \) of a point charge \( q \) at \( \vec{r}' \). Make sure that the volume integral of \( \rho \) equals \( q \).
		\item What is the volume charge desnity of an electric dipole, consisting of a point charge \( -q \) at the origin at a point charge \( +q \) at \( \vec{a} \)?
		\item What is the volume charge density (in spherical coordinates) of a uniform, infinitesimally thin spherical shell of radius \( R \) and total charge \( Q \), centered at the origin?
	\end{enumerate}

	\hwkSol
	\hwkPart
	\[
		\rho(\vec{r})=q\,\delta^3(\vec{r}-\vec{r}\,').
	\]
	\[
		\int \rho(\vec{r})\,d\tau = q.
	\]
	\hwkPart
	\[
		\rho(\vec{r})=-q\,\delta^3(\vec{r})+q\,\delta^3(\vec{r}-\vec{a}).
	\]
	\hwkPart
	\[
		\rho(r,\theta,\phi)=\frac{Q}{4\pi R^2}\,\delta(r-R).
	\]
\end{hwkProblem}

\begin{hwkProblem}{1.48}{}

	Evaluate the following integrals:
	\begin{enumerate}
		\item \(\int\left(r^2+\mathbf{r} \cdot \mathbf{a}+a^2\right) \delta^3(\mathbf{r}-\mathbf{a}) d \tau\), where \(\mathbf{a}\) is a fixed vector, \(a\) is its magnitude, and the integral is over all space.
		\item \(\int_{\mathcal{V}}|\mathbf{r}-\mathbf{b}|^2 \delta^3(5 \mathbf{r}) d \tau\), where \(\mathcal{V}\) is a cube of side 2 , centered on the origin, and \(\mathbf{b}=4 \hat{\mathbf{y}}+3 \hat{\mathbf{z}}\).
		\item \(\int_{\mathcal{V}}\left[r^4+r^2(\mathbf{r} \cdot \mathbf{c})+c^4\right] \delta^3(\mathbf{r}-\mathbf{c}) d \tau\), where \(\mathcal{V}\) is a sphere of radius 6 about the origin, \(\mathbf{c}=5 \hat{\mathbf{x}}+3 \hat{\mathbf{y}}+2 \hat{\mathbf{z}}\), and \(c\) is its magnitude.
		\item \(\int_{\mathcal{V}} \mathbf{r} \cdot(\mathbf{d}-\mathbf{r}) \delta^3(\mathbf{e}-\mathbf{r}) d \tau\), where \(\mathbf{d}=(1,2,3), \mathbf{e}=(3,2,1)\), and \(\mathcal{V}\) is a sphere of radius 1.5 centered at \((2,2,2)\).
	\end{enumerate}

	\hwkSol
	\hwkPart
	\[
		I=\int\left(r^2+\mathbf{r}\cdot\mathbf{a}+a^2\right)\delta^3(\mathbf{r}-\mathbf{a})\,d\tau
		=\Bigl(a^2+\mathbf{a}\cdot\mathbf{a}+a^2\Bigr)
		=3a^2.
	\]
	\hwkPart
	\[
		\delta^3(5\mathbf{r})=\frac{1}{5^3}\delta^3(\mathbf{r})
		=\frac{1}{125}\delta^3(\mathbf{r}).
	\]
	\[
		I=\int_{\mathcal{V}}|\mathbf{r}-\mathbf{b}|^2\,\delta^3(5\mathbf{r})\,d\tau
		=\frac{1}{125}\,|\mathbf{0}-\mathbf{b}|^2
		=\frac{b^2}{125}.
	\]
	\[
		b^2=4^2+3^2=16+9=25,\quad I=\frac{25}{125}=\frac{1}{5}.
	\]
	\hwkPart
	\[
		I=\int_{\mathcal{V}}\Bigl[r^4+r^2(\mathbf{r}\cdot\mathbf{c})+c^4\Bigr]\delta^3(\mathbf{r}-\mathbf{c})\,d\tau
		=\Bigl[c^4+c^2(\mathbf{c}\cdot\mathbf{c})+c^4\Bigr]
		=3c^4.
	\]
	\hwkPart
	\[
		I=\int_{\mathcal{V}}\mathbf{r}\cdot(\mathbf{d}-\mathbf{r})\delta^3(\mathbf{e}-\mathbf{r})\,d\tau
		=\mathbf{e}\cdot(\mathbf{d}-\mathbf{e}).
	\]
	\[
		\begin{aligned}
			\mathbf{d} &= (1,2,3),\quad \mathbf{e}=(3,2,1),\\[1mm]
			\mathbf{e}\cdot\mathbf{d} &= 3\cdot1+2\cdot2+1\cdot3=10,\\[1mm]
			\mathbf{e}\cdot\mathbf{e} &= 3^2+2^2+1^2=14,\\[1mm]
			I &= 10-14=-4.
		\end{aligned}
	\]
\end{hwkProblem}

\begin{hwkProblem}{2.1}{}

	\begin{enumerate}
		\item Twelve equal charges, \( q \), are situated at the corners of a regular 12-sided polygon (for instance, on on each numeral of a clock face). What is the net force on a test charge \( Q \) at the center?
		\item Suppose \textit{one} of the 12 \( q \)s is removed (the one at "6 o'clock"). What is the force on \( Q \)? Explain your reasoning.
		\item Now 13 equal charges, \( q \), are situated at the corners of a regular 13-sided polygon. What is the net force on a test charge \( Q \) at the center?
		\item If one of the 13 \( q \)s is removed, what is the force on \( Q \)? Explain your reasoning.
	\end{enumerate}

	\hwkSol

	\hwkPart
	\[
		\text{For 12 charges symmetrically arranged: } \vec{F}= \vec{0}.
	\]
	\hwkPart
	\[
		\text{Removing one charge: } F=\frac{k\,Qq}{R^2}\quad \text{(direction opposite to the missing charge)}.
	\]
	\hwkPart
	\[
		\text{For 13 charges symmetrically arranged: } \vec{F}= \vec{0}.
	\]
	\hwkPart
	\[
		\text{Removing one charge: } F=\frac{k\,Qq}{R^2}\quad \text{(direction opposite to the missing charge)}.
	\]
\end{hwkProblem}

\begin{hwkProblem}{2.2}{}

	Find the electric field (magnitude and direction) a distance \( z \) above the midpoint between equal and opposite charges \( \left( \pm q \right) \), a distance \( d \) apart.

	\hwkSol
	\[
		\text{Let the charges be at } \left(\pm \frac{d}{2},0,0\right).
	\]
	\[
		E_z=\frac{1}{4\pi\epsilon_0}\left[\frac{q\,z}{\left(\left(\frac{d}{2}\right)^2+z^2\right)^{3/2}}
		-\frac{(-q)\,z}{\left(\left(\frac{d}{2}\right)^2+z^2\right)^{3/2}}\right]
		=\frac{1}{4\pi\epsilon_0}\frac{2q\,z}{\left(\left(\frac{d}{2}\right)^2+z^2\right)^{3/2}}.
	\]
	\[
		\vec{E}=\frac{qz}{2\pi\epsilon_0\left[\left(\frac{d}{2}\right)^2+z^2\right]^{3/2}}\hat{z}.
	\]
\end{hwkProblem}

\end{document}
