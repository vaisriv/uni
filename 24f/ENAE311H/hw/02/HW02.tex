 \documentclass{article}

\usepackage{fancyhdr}
\usepackage{extramarks}
\usepackage{amsmath}
\usepackage{amsthm}
\usepackage{amsfonts}
\usepackage{amssymb}
\usepackage{xparse}
\usepackage{tikz}
\usepackage{graphicx}
\usepackage[plain]{algorithm}
\usepackage{algpseudocode}
\usepackage{listings}
\usepackage{hyperref}
\usepackage[per-mode = fraction]{siunitx}
\usepackage{calc}

\usetikzlibrary{automata,positioning}

\hypersetup{
    colorlinks=true,
    linkcolor=blue,
    filecolor=magenta,
    urlcolor=blue,
    }

\urlstyle{same}

%
% Basic Document Settings
%

\topmargin=-0.45in
\evensidemargin=0in
\oddsidemargin=0in
\textwidth=6.5in
\textheight=9.0in
\headsep=0.25in

\linespread{1.1}

\pagestyle{fancy}
\lhead{\hmwkAuthorName}
\chead{\hmwkClass\ (\hmwkClassInstructor,\ \hmwkClassTime): \hmwkTitle}
\rhead{\firstxmark}
\lfoot{\lastxmark}
\cfoot{\thepage}

\renewcommand\headrulewidth{0.4pt}
\renewcommand\footrulewidth{0.4pt}

\setlength\parindent{0pt}
\allowdisplaybreaks
%
% Title Page
%

\title{
	\vspace{2in}
	\textmd{\textbf{\hmwkClass:\ \hmwkTitle}}\\
	\normalsize\vspace{0.1in}\small{Due\ on\ \hmwkDueDate\ at \hmwkDueTime}\\
	\vspace{0.1in}\large{\textit{\hmwkClassInstructor,\ \hmwkClassTime}}
	\vspace{3in}
}
\author{\textbf{\hmwkAuthorName}}
\date{\hmwkCompletionDate}

%
% Create Problem Sections
%

\newcommand{\enterProblemHeader}[1]{
	\nobreak\extramarks{}{Problem #1 continued on next page\ldots}\nobreak{}
	\nobreak\extramarks{Problem #1 (continued)}{Problem #1 continued on next page\ldots}\nobreak{}
}

\newcommand{\exitProblemHeader}[1]{
	\nobreak\extramarks{Problem #1 (continued)}{Problem #1 continued on next page\ldots}\nobreak{}
	\nobreak\extramarks{Problem #1}{}\nobreak{}
}

%
% Homework Problem Environment
%
\NewDocumentEnvironment{hwkProblem}{m m s}{
	\section*{Problem #1: #2}
	\enterProblemHeader{#1}
	\setcounter{partCounter}{1}
}{
	\exitProblemHeader{#1}
	\IfBooleanF{#3} % if star, no new page
		{\newpage}
}

% Alias for the Solution section header
\newcommand{\hwkSol}{\vspace{\baselineskip / 2}\textbf{\Large Solution}\vspace{\baselineskip / 2}}

% Alias for the Solution Part subsection header
\newcounter{partCounter}
\newcommand{\hwkPart}{
	\vspace{\baselineskip / 2}
	\textbf{\large Part \Alph{partCounter}}
	\vspace{\baselineskip / 2}
	\stepcounter{partCounter}
}

%
% Various Helper Commands
%

% Such That
\newcommand{\st}{\text{s.t.}}

% Useful for algorithms
\newcommand{\alg}[1]{\textsc{\bfseries \footnotesize #1}}

% For derivatives
\newcommand{\deriv}[1]{\frac{\mathrm{d}}{\mathrm{d}x} (#1)}

% For partial derivatives
\newcommand{\pderiv}[2]{\frac{\partial}{\partial #1} (#2)}

% Integral dx
\newcommand{\dx}{\mathrm{d}x}
\newcommand{\dy}{\mathrm{d}y}

% Probability commands: Expectation, Variance, Covariance, Bias
\newcommand{\e}[1]{\mathrm{e}#1}
\newcommand{\E}{\mathrm{E}}
\newcommand{\Var}{\mathrm{Var}}
\newcommand{\Cov}{\mathrm{Cov}}
\newcommand{\Bias}{\mathrm{Bias}}

% Defining Units that are not in the SI base
\DeclareSIUnit\bar{bar}
\DeclareSIUnit\ft{ft}
\DeclareSIUnit\dollar{\$}
\DeclareSIUnit\cent{\text{\textcent}}
\DeclareSIUnit\c{\degreeCelsius}

% Code Listing config
\usepackage{xcolor}
\definecolor{codegreen}{rgb}{0,0.6,0}
\definecolor{codegray}{rgb}{0.5,0.5,0.5}
\definecolor{codepurple}{rgb}{0.58,0,0.82}
\definecolor{backcolour}{rgb}{0.95,0.95,0.92}
\lstdefinestyle{overleaf}{
	% backgroundcolor=\color{backcolour},
	commentstyle=\color{codegreen},
	keywordstyle=\color{magenta},
	numberstyle=\tiny\color{codegray},
	stringstyle=\color{codepurple},
	basicstyle=\ttfamily\footnotesize,
	breakatwhitespace=false,
	breaklines=true,
	captionpos=b,
	keepspaces=true,
	numbers=left,
	numbersep=5pt,
	showspaces=false,
	showstringspaces=false,
	showtabs=false,
	tabsize=4
}

\usepackage[latte]{catppuccinpalette}
\lstdefinestyle{catppuccin}{
	breaklines=true,
	keepspaces=true,
	numbers=left,
	numbersep=5pt,
	showspaces=false,
	showstringspaces=false,
	breakatwhitespace=true,
	tabsize=4,
	stringstyle = {\color{CtpGreen}},
	commentstyle={\color{CtpOverlay1}},
	basicstyle = {\small\color{CtpText}\ttfamily},
	keywordstyle = {\color{CtpMauve}},
	keywordstyle = [2]{\color{CtpBlue}},
	keywordstyle = [3]{\color{CtpYellow}},
	keywordstyle = [4]{\color{CtpLavender}},
	keywordstyle = [5]{\color{CtpPeach}},
	keywordstyle = [6]{\color{CtpTeal}}
}

\lstset{style=catppuccin}


%
% Homework Details
%   - Title
%   - Due date
%   - Due time
%   - Course
%   - Section/Time
%   - Instructor
%   - Author
%

\newcommand{\hmwkTitle}{Homework 02}
\newcommand{\hmwkDueDate}{September 27, 2024}
\newcommand{\hmwkDueTime}{05:00 PM}
\newcommand{\hmwkClass}{ENAE 311H}
\newcommand{\hmwkClassTime}{Section 0101}
\newcommand{\hmwkClassInstructor}{Dr. Brehm}
\newcommand{\hmwkAuthorName}{\textbf{Vai Srivastava}}
\newcommand{\hmwkCompletionDate}{September 27, 2024}

\begin{document}

\maketitle

\pagebreak

\begin{homeworkProblem}

	Mach number: \(M = 3.6\),
	Altitude temperature: \(T = 221 \, \text{K}\).

	\[
		a = \sqrt{1.4 \times 287 \times 221} \, \text{m/s}
	\]
	\[
		V = M \times a
	\]

	Free-stream temperature: \(T_\infty = 120 \, \text{K}\),
	Free-stream velocity: \(V_\infty = 790.5 \, \text{m/s}\).

	\[
		a_\infty = \sqrt{1.4 \times 287 \times 120} \, \text{m/s}
	\]

	\[
		M_\infty = \frac{V_\infty}{a_\infty}
	\]

	\[
		Re = \frac{\rho V L}{\mu}
	\]

	\[
		Re_{\text{SR-71}} = Re_{\text{model}}
	\]

	Speed of sound for the SR-71: \(297.99 \, \text{m/s}\)

	Velocity of the SR-71: \(1072.76 \, \text{m/s}\)

	Mach number in the wind tunnel: \(3.60\)

	Speed of sound in the wind tunnel: \(219.58 \, \text{m/s}\)


	The Mach number in both the SR-71 flight and the wind tunnel experiment is essentially the same (\(M = 3.6\)), meaning that \textbf{the flows are Mach-similar}.


	Reynolds number for the SR-71: \(0.86\)
	Reynolds number for the wind tunnel: \(0.65\)
	Viscosity ratio (SR-71 to wind tunnel): \(1.50\)


	Since the Reynolds numbers are not equal, the flows are \textbf{not dynamically similar}.

\end{homeworkProblem}

\begin{homeworkProblem}
	\begin{align*}
		\text{Given data:}                                                                                                                                                 \\
		V_{\text{flight}} = 400 \, \text{m/s}, \, T_{\text{flight}} = 217 \, \text{K}, \, \rho_{\text{flight}} = 0.30 \, \text{kg/m}^3                                     \\
		P_{\text{wind}} = 75000 \, \text{Pa}, \, \text{L}_{\text{scale}} = \frac{1}{5}                                                                                     \\

		\text{Step 1: Calculate speed of sound in flight:}                                                                                                                 \\
		a_{\text{flight}} = \sqrt{\gamma R T_{\text{flight}}}                                                                                                              \\
		a_{\text{flight}} = \sqrt{1.4 \times 287 \times 217} = 295.28 \, \text{m/s}                                                                                        \\

		\text{Step 2: Calculate Mach number in flight:}                                                                                                                    \\
		M_{\text{flight}} = \frac{V_{\text{flight}}}{a_{\text{flight}}} = \frac{400}{295.28} = 1.3546                                                                      \\

		\text{Step 3: Calculate speed of sound in wind tunnel based on guessed temperature } T_{\text{wind}} = 300 \, \text{K}:                                            \\
		a_{\text{wind}} = \sqrt{\gamma R T_{\text{wind}}}                                                                                                                  \\
		a_{\text{wind}} = \sqrt{1.4 \times 287 \times 300} = 347.19 \, \text{m/s}                                                                                          \\

		\text{Step 4: Calculate required velocity for Mach similarity in wind tunnel:}                                                                                     \\
		V_{\text{wind}} = M_{\text{flight}} \times a_{\text{wind}} = 1.3546 \times 347.19 = 470.32 \, \text{m/s}                                                           \\

		\text{Step 5: Calculate density in wind tunnel using ideal gas law:}                                                                                               \\
		\rho_{\text{wind}} = \frac{P_{\text{wind}}}{R T_{\text{wind}}} = \frac{75000}{287 \times 300} = 0.871 \, \text{kg/m}^3                                             \\

		\text{Step 6: Calculate Reynolds numbers:}                                                                                                                         \\
		\mu \propto T^{2/3}                                                                                                                                                \\
		\mu_{\text{ratio}} = \frac{T_{\text{flight}}^{2/3}}{T_{\text{wind}}^{2/3}} = \frac{217^{2/3}}{300^{2/3}} = 0.8058                                                  \\

		\text{Reynolds number for flight:}                                                                                                                                 \\
		Re_{\text{flight}} = \frac{\rho_{\text{flight}} V_{\text{flight}} L_{\text{flight}}}{\mu_{\text{flight}}} = \frac{0.30 \times 400 \times 1}{217^{2/3}} = 3.323     \\

		\text{Reynolds number for wind tunnel (with L = 1/5):}                                                                                                             \\
		Re_{\text{wind}} = \frac{\rho_{\text{wind}} V_{\text{wind}} L_{\text{wind}}}{\mu_{\text{wind}}} = \frac{0.871 \times 470.32 \times \frac{1}{5}}{300^{2/3}} = 1.828 \\

		\text{Step 7: Solve for temperature to match Reynolds number similarity:}                                                                                          \\
		Re_{\text{flight}} = Re_{\text{wind}}                                                                                                                              \\
		\Rightarrow \frac{\rho_{\text{flight}} V_{\text{flight}}}{T_{\text{flight}}^{2/3}} = \frac{\rho_{\text{wind}} V_{\text{wind}}}{T_{\text{wind}}^{2/3}}              \\
		T_{\text{wind}} = 179.77 \, \text{K}                                                                                                                               \\

		\text{Final calculations:}                                                                                                                                         \\
		T_{\text{wind}} = 179.77 \, \text{K}                                                                                                                               \\
		V_{\text{wind final}} = 1.3546 \times 268.83 = 364.07 \, \text{m/s}                                                                                                \\
		\rho_{\text{wind final}} = \frac{75000}{287 \times 179.77} = 1.454 \, \text{kg/m}^3                                                                                \\
		\end{align*}
\end{homeworkProblem}

\end{document}
