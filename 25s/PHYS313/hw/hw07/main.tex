\documentclass{article}

\usepackage{fancyhdr}
\usepackage{extramarks}
\usepackage{amsmath}
\usepackage{amsthm}
\usepackage{amsfonts}
\usepackage{amssymb}
\usepackage{xparse}
\usepackage{tikz}
\usepackage{graphicx}
\usepackage[plain]{algorithm}
\usepackage{algpseudocode}
\usepackage{listings}
\usepackage{hyperref}
\usepackage[per-mode = fraction]{siunitx}
\usepackage{calc}

\usetikzlibrary{automata,positioning}

\hypersetup{
    colorlinks=true,
    linkcolor=blue,
    filecolor=magenta,
    urlcolor=blue,
    }

\urlstyle{same}

%
% Basic Document Settings
%

\topmargin=-0.45in
\evensidemargin=0in
\oddsidemargin=0in
\textwidth=6.5in
\textheight=9.0in
\headsep=0.25in

\linespread{1.1}

\pagestyle{fancy}
\lhead{\hmwkAuthorName}
\chead{\hmwkClass\ (\hmwkClassInstructor,\ \hmwkClassTime): \hmwkTitle}
\rhead{\firstxmark}
\lfoot{\lastxmark}
\cfoot{\thepage}

\renewcommand\headrulewidth{0.4pt}
\renewcommand\footrulewidth{0.4pt}

\setlength\parindent{0pt}
\allowdisplaybreaks
%
% Title Page
%

\title{
	\vspace{2in}
	\textmd{\textbf{\hmwkClass:\ \hmwkTitle}}\\
	\normalsize\vspace{0.1in}\small{Due\ on\ \hmwkDueDate\ at \hmwkDueTime}\\
	\vspace{0.1in}\large{\textit{\hmwkClassInstructor,\ \hmwkClassTime}}
	\vspace{3in}
}
\author{\textbf{\hmwkAuthorName}}
\date{\hmwkCompletionDate}

%
% Create Problem Sections
%

\newcommand{\enterProblemHeader}[1]{
	\nobreak\extramarks{}{Problem #1 continued on next page\ldots}\nobreak{}
	\nobreak\extramarks{Problem #1 (continued)}{Problem #1 continued on next page\ldots}\nobreak{}
}

\newcommand{\exitProblemHeader}[1]{
	\nobreak\extramarks{Problem #1 (continued)}{Problem #1 continued on next page\ldots}\nobreak{}
	\nobreak\extramarks{Problem #1}{}\nobreak{}
}

%
% Homework Problem Environment
%
\NewDocumentEnvironment{hwkProblem}{m m s}{
	\section*{Problem #1: #2}
	\enterProblemHeader{#1}
	\setcounter{partCounter}{1}
}{
	\exitProblemHeader{#1}
	\IfBooleanF{#3} % if star, no new page
		{\newpage}
}

% Alias for the Solution section header
\newcommand{\hwkSol}{\vspace{\baselineskip / 2}\textbf{\Large Solution}\vspace{\baselineskip / 2}}

% Alias for the Solution Part subsection header
\newcounter{partCounter}
\newcommand{\hwkPart}{
	\vspace{\baselineskip / 2}
	\textbf{\large Part \Alph{partCounter}}
	\vspace{\baselineskip / 2}
	\stepcounter{partCounter}
}

%
% Various Helper Commands
%

% Such That
\newcommand{\st}{\text{s.t.}}

% Useful for algorithms
\newcommand{\alg}[1]{\textsc{\bfseries \footnotesize #1}}

% For derivatives
\newcommand{\deriv}[1]{\frac{\mathrm{d}}{\mathrm{d}x} (#1)}

% For partial derivatives
\newcommand{\pderiv}[2]{\frac{\partial}{\partial #1} (#2)}

% Integral dx
\newcommand{\dx}{\mathrm{d}x}
\newcommand{\dy}{\mathrm{d}y}

% Probability commands: Expectation, Variance, Covariance, Bias
\newcommand{\e}[1]{\mathrm{e}#1}
\newcommand{\E}{\mathrm{E}}
\newcommand{\Var}{\mathrm{Var}}
\newcommand{\Cov}{\mathrm{Cov}}
\newcommand{\Bias}{\mathrm{Bias}}

% Defining Units that are not in the SI base
\DeclareSIUnit\bar{bar}
\DeclareSIUnit\ft{ft}
\DeclareSIUnit\dollar{\$}
\DeclareSIUnit\cent{\text{\textcent}}
\DeclareSIUnit\c{\degreeCelsius}

% Code Listing config
\usepackage{xcolor}
\definecolor{codegreen}{rgb}{0,0.6,0}
\definecolor{codegray}{rgb}{0.5,0.5,0.5}
\definecolor{codepurple}{rgb}{0.58,0,0.82}
\definecolor{backcolour}{rgb}{0.95,0.95,0.92}
\lstdefinestyle{overleaf}{
	% backgroundcolor=\color{backcolour},
	commentstyle=\color{codegreen},
	keywordstyle=\color{magenta},
	numberstyle=\tiny\color{codegray},
	stringstyle=\color{codepurple},
	basicstyle=\ttfamily\footnotesize,
	breakatwhitespace=false,
	breaklines=true,
	captionpos=b,
	keepspaces=true,
	numbers=left,
	numbersep=5pt,
	showspaces=false,
	showstringspaces=false,
	showtabs=false,
	tabsize=4
}

\usepackage[latte]{catppuccinpalette}
\lstdefinestyle{catppuccin}{
	breaklines=true,
	keepspaces=true,
	numbers=left,
	numbersep=5pt,
	showspaces=false,
	showstringspaces=false,
	breakatwhitespace=true,
	tabsize=4,
	stringstyle = {\color{CtpGreen}},
	commentstyle={\color{CtpOverlay1}},
	basicstyle = {\small\color{CtpText}\ttfamily},
	keywordstyle = {\color{CtpMauve}},
	keywordstyle = [2]{\color{CtpBlue}},
	keywordstyle = [3]{\color{CtpYellow}},
	keywordstyle = [4]{\color{CtpLavender}},
	keywordstyle = [5]{\color{CtpPeach}},
	keywordstyle = [6]{\color{CtpTeal}}
}

\lstset{style=catppuccin}


%
% Homework Details
%   - Title
%   - Subtitle
%   - Due date
%   - Due time
%   - Course
%   - Section/Time
%   - Instructor
%   - Author
%

\newcommand{\hmwkTitle}{HW 07}
\newcommand{\hmwkSubTitle}{Assignment 7}
\newcommand{\hmwkDueDate}{March 27th, 2025}
\newcommand{\hmwkDueTime}{11:59 PM}
\newcommand{\hmwkClass}{PHYS 313}
\newcommand{\hmwkClassTime}{0101}
\newcommand{\hmwkClassInstructor}{Dr.\ Ji}
\newcommand{\hmwkAuthorName}{\textbf{Vai Srivastava}}
\newcommand{\hmwkCompletionDate}{\today}

\begin{document}

\maketitle

\pagebreak

\begin{hwkProblem}{3.15}{}

	A rectangular pipe, running parallel to the \( z \)-axis (from \( - \infty \) to \( + \infty \), has three grounded metal sides, at \( y = 0 \), \( y = a \), and \( x = 0 \). The fourth side, at \( x = b \), is maintained at a specific potential \( \func{V_{0}}[y] \).

	\hwkSol{}

	\begin{align*}
		\derivsec{x}{\func{V}[x, y]} + \derivsec{y}{\func{V}[x, y]} &= 0 \\
		\text{boundary} &= \begin{cases}
			i:& \func{V}[x, 0] = 0 \\
			ii:& \func{V}[x, a] = 0 \\
			iii:& \func{V}[0, y] = 0 \\
			iv:& \func{V}[b, y] = \func{V_{0}}[y]
		\end{cases} \\
		\func{V}[x, y] &= \left(A e^{kx} + B e^{-kx}\right)\left(C \sin{ky} + D \cos{ky}\right) \\
		i &\implies D = 0 \\
		ii &\implies ka = n \pi, n \in \mathbb{Z} \\
		iii &\implies B = -A \\
		\func{V}[x, y] &= AC \left(e^{n\pi \frac{x}{a}} - e^{-n \pi \frac{x}{a}}\right)\sin{n \pi \frac{y}{a}} \\
			       &= 2AC\sinh{n \pi \frac{x}{a}}\sin{n \pi \frac{y}{a}} \\
			       &= \frac{4 V_{0}}{\pi} \sum_{n=1, 3, 5, \dots} \frac{\sinh{n \pi \frac{x}{a}} \sin{n \pi \frac{y}{a}}}{n \sinh{n \pi \frac{b}{a}}} \\
	\end{align*}
	\center{\boxed{\func{V}[x, y] = \frac{4 V_{0}}{\pi} \sum_{n = 1}^{\infty} \frac{\sinh{\left(2n-1\right) \pi \frac{x}{a}} \sin{\left(2n-1\right) \pi \frac{y}{a}}}{\left(2n-1\right) \sinh{\left(2n-1\right) \pi \frac{b}{a}}}}}

\end{hwkProblem}
\begin{hwkProblem}{3.17}{}

	Derive \( \func{P_{3}}[x] \) from the Rodrigues formula, and check that \( \func{P_{3}}[\cos{\theta}] \) satisfiest the angular equation (3.60) for \( l=3 \). Check that \( P_{3} \) and \( P_{1} \) are orthogonal by explicit integration.

	\hwkSol{}

	\begin{align*}
		P_l(x) &= \frac{1}{2^l l!}\frac{d^l}{dx^l} (x^2-1)^l \\
		P_3(x) &= \frac{1}{2^3 3!}\frac{d^3}{dx^3} (x^2-1)^3 \\
		(x^2-1)^3 &= x^6 - 3x^4 + 3x^2 - 1 \\
		\frac{d}{dx}(x^6-3x^4+3x^2-1) &= 6x^5 - 12x^3 + 6x \\
		\frac{d^2}{dx^2}(x^6-3x^4+3x^2-1) &= 30x^4 - 36x^2 + 6 \\
		\frac{d^3}{dx^3}(x^6-3x^4+3x^2-1) &= 120x^3 - 72x \\
		P_3(x) &= \frac{1}{2^3 3!}(120x^3-72x) = \frac{1}{8\cdot6}(120x^3-72x)=\frac{1}{48}(120x^3-72x) \\
		P_3(x) &= \frac{120}{48}x^3 - \frac{72}{48}x = \frac{5}{2}x^3 - \frac{3}{2}x \\
		0 &= \frac{1}{\sin{\theta}}\frac{d}{d\theta}\left(\sin{\theta} \frac{d}{d\theta}P_3(\cos{\theta})\right) + l(l+1) P_3(\cos{\theta}), \\
		y(\theta) &= P_3(\cos{\theta})=\frac{5}{2}\cos{\theta}^3-\frac{3}{2}\cos{\theta} \\
		\frac{dy}{d\theta} &= \frac{5}{2}\cdot 3\cos{\theta}^2\,(-\sin{\theta}) - \frac{3}{2}(-\sin{\theta}) = -\frac{15}{2}\cos{\theta}^2\,\sin{\theta}+\frac{3}{2}\sin{\theta} \\
		\frac{dy}{d\theta} &= \frac{3\sin{\theta}}{2}\left(1-5\cos{\theta}^2\right) \\
		\sin{\theta}\frac{dy}{d\theta} &= \frac{3\sin{}^2\theta}{2}\left(1-5\cos{}^2\theta\right) \\
		A(\theta) &= \sin{\theta}\frac{dy}{d\theta} = \frac{3}{2}\sin{\theta}^2\left(1-5\cos{\theta}^2\right) \\
		\frac{dA}{d\theta} &= \frac{3}{2}\left[ 2\sin{\theta}\cos{\theta}\,(1-5\cos{\theta}^2) + \sin{\theta}^2\,(10\cos{\theta}\sin{\theta}) \right] \\
		\frac{dA}{d\theta} &= \frac{3}{2}\left[ 2\sin{\theta}\cos{\theta} (1-5\cos{\theta}^2) + 10\sin{\theta}^3\cos{\theta} \right] \\
		\frac{dA}{d\theta} &= \frac{3}{2}\cdot 2\sin{\theta}\cos{\theta} \left[(1-5\cos{\theta}^2)+5\sin{\theta}^2\right] \\
		(1-5\cos{\theta}^2)+5(1-\cos{\theta}^2) &= 1-5\cos{\theta}^2+5-5\cos{\theta}^2 = 6-10\cos{\theta}^2 \\
		\frac{dA}{d\theta} &= 3\sin{\theta}\cos{\theta} \left(6-10\cos{\theta}^2\right) \\
		\frac{1}{\sin{\theta}}\frac{d}{d\theta}\left(\sin{\theta}\frac{dy}{d\theta}\right) &= 3\cos{\theta}\left(6-10\cos{\theta}^2\right) = 18\cos{\theta}-30\cos{\theta}^3 \\
		12y(\theta) &= 12\left(\frac{5}{2}\cos{\theta}^3-\frac{3}{2}\cos{\theta}\right) = 30\cos{\theta}^3-18\cos{\theta} \\
		0 &= \left[18\cos{\theta}-30\cos{\theta}^3\right] + \left[30\cos{\theta}^3-18\cos{\theta}\right] \qed \\
		\int_{-1}^{1} P_l(x) P_{l'}(x) \, dx &= 0 \quad \text{for } l\neq l' \\
		P_3(x) &= \frac{5}{2}x^3-\frac{3}{2}x \quad \text{and} \quad P_1(x)= x \\
		P_3(x)P_1(x) &= \left(\frac{5}{2}x^3-\frac{3}{2}x\right)x = \frac{5}{2}x^4-\frac{3}{2}x^2 \\
		\int_{-1}^{1} \left(\frac{5}{2}x^4-\frac{3}{2}x^2\right)dx &= \frac{5}{2}\int_{-1}^{1} x^4\,dx - \frac{3}{2}\int_{-1}^{1} x^2\,dx \\
		\int_{-1}^{1} x^{2n}\,dx &= \frac{2}{2n+1} \\
		\int_{-1}^{1} x^4\,dx &= \frac{2}{5}, \quad \int_{-1}^{1} x^2\,dx = \frac{2}{3} \\
		\frac{5}{2}\cdot \frac{2}{5} - \frac{3}{2}\cdot \frac{2}{3} &= 1 - 1 = 0 \qed \\
	\end{align*}
	\center{\boxed{P_3(x) = \frac{5}{2}x^3 - \frac{3}{2}x}}
	\center{\boxed{\int_{-1}^{1} P_3(x) P_1(x) \, dx = 0}}

\end{hwkProblem}
\begin{hwkProblem}{3.18}{}

	\begin{enumerate}
		\item Suppose the potential is a \textit{constant} \( V_{0} \) over the surface of the sphere. Use the results of Ex:3.6 and Ex:3.7 to find the potential inside and outside the sphere.
		\item Find the potential inside and outside a spherical shell that carries a uniform surface charge \( \sigma_{0} \), using the results of Ex:3.9.
	\end{enumerate}

	\hwkSol{}

	\hwkPart{}
	\begin{align*}
		V(R) = V_0, \\
		\text{Inside (\(r\le R\)):} \quad &V_{\text{in}}(r)=V_0, \\
		\text{Outside (\(r\ge R\)):} \quad &V_{\text{out}}(r)=\frac{V_0\,R}{r}
	\end{align*}

	\hwkPart{}
	\begin{align*}
		\sigma_0\text{ on }r=R, Q=4\pi R^2\sigma_0, \\
		\text{Outside (\(r\ge R\)):} \quad &V_{\text{out}}(r)=\frac{1}{4\pi\epsilon_0}\frac{Q}{r}=\frac{R^2\sigma_0}{\epsilon_0\,r}, \\
		\text{Inside (\(r\le R\)):} \quad &V_{\text{in}}(r)=V(R)=\frac{R^2\sigma_0}{\epsilon_0\,R}=\frac{R\sigma_0}{\epsilon_0}
	\end{align*}

\end{hwkProblem}
\begin{hwkProblem}{3.20}{}

	Suppose the potential \( \func{V_{0}}[\theta] \) at the surface of a sphere is specified, and there is no charge inside or outside the sphere. Show that the charge density on the sphere is given by
	\[
		\func{\sigma}[\theta] = \frac{\epsilon_{0}}{2 R} \sum_{l=0}^{\infty}\left(2l+1\right)^{2}\func{C_{l}}\func{P_{l}}[\cos{\theta}]
	\]
	where
	\[
		\func{C_{l}} = \intdef{0}{\pi}{\func{V_{0}}[\theta]\func{P_{l}}[\cos{\theta}]\sin{\theta}}{\theta}
	\]

	\hwkSol{}

	\begin{align*}
		V_{\text{in}}(r,\theta) &= \sum_{l=0}^{\infty} A_{l}\, r^{\,l} P_{l}(\cos\theta), \\
		V_{\text{out}}(r,\theta) &= \sum_{l=0}^{\infty} B_{l}\, r^{-(l+1)} P_{l}(\cos\theta) \\
		V_{0}(\theta) &= V_{\text{in}}(R,\theta) = V_{\text{out}}(R,\theta) \\
		&=\sum_{l=0}^{\infty} A_{l}\, R^{l} P_{l}(\cos\theta) \\
		A_{l}R^{l} &= \frac{2l+1}{2} \int_{0}^{\pi}V_{0}(\theta) P_{l}(\cos\theta) \sin\theta\, d\theta \equiv \frac{2l+1}{2}\, C_{l}, \\
		\Rightarrow\quad A_{l} &= \frac{2l+1}{2}\, \frac{C_{l}}{R^{l}} \\
		C_{l} &= \int_{0}^{\pi}V_{0}(\theta) P_{l}(\cos\theta) \sin\theta\, d\theta \\
		\sigma(\theta) &= \epsilon_{0}\left[-\left.\frac{\partial V_{\text{out}}}{\partial r}\right|_{r=R} + \left.\frac{\partial V_{\text{in}}}{\partial r}\right|_{r=R}\right] \\
		\left.\frac{\partial V_{\text{in}}}{\partial r}\right|_{r=R} &= \sum_{l=0}^{\infty} l\, A_{l}\, R^{l-1} P_{l}(\cos\theta), \\
		\left.\frac{\partial V_{\text{out}}}{\partial r}\right|_{r=R} &= -\sum_{l=0}^{\infty} (l+1)\, B_{l}\, R^{-(l+2)} P_{l}(\cos\theta) \\
		& A_{l}\, R^{l} = B_{l}\, R^{-(l+1)} \\
		& B_{l} = A_{l}\, R^{2l+1} \\
		\left.\frac{\partial V_{\text{out}}}{\partial r}\right|_{r=R} 
		&= -\sum_{l=0}^{\infty} (l+1)\, A_{l}\, R^{2l+1}\, R^{-(l+2)} P_{l}(\cos\theta) \\
		&= -\sum_{l=0}^{\infty} (l+1)\, A_{l}\, R^{l-1} P_{l}(\cos\theta) \\
		-\left.\frac{\partial V_{\text{out}}}{\partial r}\right|_{r=R} + \left.\frac{\partial V_{\text{in}}}{\partial r}\right|_{r=R} &=\sum_{l=0}^{\infty} \left[(l+1) + l\right] A_{l}\, R^{l-1} P_{l}(\cos\theta) \\
		&=\sum_{l=0}^{\infty} (2l+1) A_{l}\, R^{l-1} P_{l}(\cos\theta) \\
		\sigma(\theta) &= \epsilon_{0} \sum_{l=0}^{\infty} (2l+1) \left(\frac{2l+1}{2}\frac{C_{l}}{R^{l}}\right) R^{l-1} P_{l}(\cos\theta) \\
		&=\frac{\epsilon_{0}}{2R}\sum_{l=0}^{\infty} (2l+1)^2\, C_{l}\, P_{l}(\cos\theta)
	\end{align*}
	\center{ \boxed{\sigma(\theta) = \frac{\epsilon_{0}}{2R} \sum_{l=0}^{\infty} (2l+1)^2\, C_{l}\, P_{l}(\cos\theta),\quad \text{with}\quad C_{l} = \int_{0}^{\pi} V_{0}(\theta) P_{l}(\cos\theta) \sin\theta\, d\theta.}}
\end{hwkProblem}
\begin{hwkProblem}{3.21}{}

	Find the potential outside a \textit{charged} metal sphere (charge \( Q \), radius \( R \)) placed in an otherwise uniform electric field \( \vecb{E}_{0} \). Explain clearly where you are setting the zero of potential.

	\hwkSol{}

	Set the zero of the potential at infinity: \(V(\infty)=0\).

	\begin{align*}
		V(\infty) &= 0, \\
		V(r,\theta) &= \frac{Q}{4\pi\epsilon_0\,r} - E_0\,r\cos\theta + E_0\,\frac{R^3}{r^2}\cos\theta,\quad r\geq R, \\
		V(R,\theta) &= \frac{Q}{4\pi\epsilon_0\,R} - E_0\,R\cos\theta + E_0\,\frac{R^3}{R^2}\cos\theta
	\end{align*}
	\center{\boxed{\func{V}[R, \theta] = \frac{Q}{4\pi\epsilon_0\,R}}}

\end{hwkProblem}
\begin{hwkProblem}{3.24}{}

	Solve Laplace's equation by separation of variables in \textit{cylindrical} coordinates, assuming there is no dependence on \( z \) (cylindrical symmetry). [Make sure you find \textit{all} solutions to the radial equation; in particular, your result must accommodate the case of an infinite line charge, for which (of course) we already know the answer.]

	\hwkSol{}

	\begin{align*}
		\nabla^2 V &= \frac{1}{r}\frac{\partial}{\partial r}\Bigl(r\frac{\partial V}{\partial r}\Bigr) + \frac{1}{r^2}\frac{\partial^2 V}{\partial \phi^2} = 0, \\
		V(r,\phi) &= R(r)\,\Phi(\phi), \\
		\frac{1}{R}\frac{d}{dr}\Bigl(r\frac{dR}{dr}\Bigr) + \frac{1}{\Phi}\frac{d^2\Phi}{d\phi^2} &= 0 
		\quad \Longrightarrow \quad r^2\frac{R''}{R}+r\frac{R'}{R} + \frac{\Phi''}{\Phi} = 0, \\
		\frac{\Phi''}{\Phi} &= -m^2 
		\quad \Longrightarrow \quad \Phi(\phi) = A\cos{m\phi}+B\sin{m\phi}, \quad m=0,1,2,\ldots, \\
		r^2R'' + rR' - m^2 R &= 0, \\
		\text{For } m\neq 0:\quad R(r) &= C\,r^m + D\,r^{-m}, \\
		\text{and for } m=0:\quad R(r) &= C_0 + D_0\ln r
	\end{align*}
	\center{\boxed{\therefore\quad V(r,\phi) = \Bigl(C_0 + D_0\ln r\Bigr) + \sum_{m=1}^{\infty}\Bigl[\Bigl(C_m\,r^m + D_m\,r^{-m}\Bigr)\cos{m\phi} + \Bigl(E_m\,r^m + F_m\,r^{-m}\Bigr)\sin{m\phi}\Bigr]}}

\end{hwkProblem}
\end{document}
