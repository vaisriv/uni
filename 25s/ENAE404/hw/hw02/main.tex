\documentclass{article}

\usepackage{fancyhdr}
\usepackage{extramarks}
\usepackage{amsmath}
\usepackage{amsthm}
\usepackage{amsfonts}
\usepackage{amssymb}
\usepackage{xparse}
\usepackage{tikz}
\usepackage{graphicx}
\usepackage[plain]{algorithm}
\usepackage{algpseudocode}
\usepackage{listings}
\usepackage{hyperref}
\usepackage[per-mode = fraction]{siunitx}
\usepackage{calc}

\usetikzlibrary{automata,positioning}

\hypersetup{
    colorlinks=true,
    linkcolor=blue,
    filecolor=magenta,
    urlcolor=blue,
    }

\urlstyle{same}

%
% Basic Document Settings
%

\topmargin=-0.45in
\evensidemargin=0in
\oddsidemargin=0in
\textwidth=6.5in
\textheight=9.0in
\headsep=0.25in

\linespread{1.1}

\pagestyle{fancy}
\lhead{\hmwkAuthorName}
\chead{\hmwkClass\ (\hmwkClassInstructor,\ \hmwkClassTime): \hmwkTitle}
\rhead{\firstxmark}
\lfoot{\lastxmark}
\cfoot{\thepage}

\renewcommand\headrulewidth{0.4pt}
\renewcommand\footrulewidth{0.4pt}

\setlength\parindent{0pt}
\allowdisplaybreaks
%
% Title Page
%

\title{
	\vspace{2in}
	\textmd{\textbf{\hmwkClass:\ \hmwkTitle}}\\
	\normalsize\vspace{0.1in}\small{Due\ on\ \hmwkDueDate\ at \hmwkDueTime}\\
	\vspace{0.1in}\large{\textit{\hmwkClassInstructor,\ \hmwkClassTime}}
	\vspace{3in}
}
\author{\textbf{\hmwkAuthorName}}
\date{\hmwkCompletionDate}

%
% Create Problem Sections
%

\newcommand{\enterProblemHeader}[1]{
	\nobreak\extramarks{}{Problem #1 continued on next page\ldots}\nobreak{}
	\nobreak\extramarks{Problem #1 (continued)}{Problem #1 continued on next page\ldots}\nobreak{}
}

\newcommand{\exitProblemHeader}[1]{
	\nobreak\extramarks{Problem #1 (continued)}{Problem #1 continued on next page\ldots}\nobreak{}
	\nobreak\extramarks{Problem #1}{}\nobreak{}
}

%
% Homework Problem Environment
%
\NewDocumentEnvironment{hwkProblem}{m m s}{
	\section*{Problem #1: #2}
	\enterProblemHeader{#1}
	\setcounter{partCounter}{1}
}{
	\exitProblemHeader{#1}
	\IfBooleanF{#3} % if star, no new page
		{\newpage}
}

% Alias for the Solution section header
\newcommand{\hwkSol}{\vspace{\baselineskip / 2}\textbf{\Large Solution}\vspace{\baselineskip / 2}}

% Alias for the Solution Part subsection header
\newcounter{partCounter}
\newcommand{\hwkPart}{
	\vspace{\baselineskip / 2}
	\textbf{\large Part \Alph{partCounter}}
	\vspace{\baselineskip / 2}
	\stepcounter{partCounter}
}

%
% Various Helper Commands
%

% Such That
\newcommand{\st}{\text{s.t.}}

% Useful for algorithms
\newcommand{\alg}[1]{\textsc{\bfseries \footnotesize #1}}

% For derivatives
\newcommand{\deriv}[1]{\frac{\mathrm{d}}{\mathrm{d}x} (#1)}

% For partial derivatives
\newcommand{\pderiv}[2]{\frac{\partial}{\partial #1} (#2)}

% Integral dx
\newcommand{\dx}{\mathrm{d}x}
\newcommand{\dy}{\mathrm{d}y}

% Probability commands: Expectation, Variance, Covariance, Bias
\newcommand{\e}[1]{\mathrm{e}#1}
\newcommand{\E}{\mathrm{E}}
\newcommand{\Var}{\mathrm{Var}}
\newcommand{\Cov}{\mathrm{Cov}}
\newcommand{\Bias}{\mathrm{Bias}}

% Defining Units that are not in the SI base
\DeclareSIUnit\bar{bar}
\DeclareSIUnit\ft{ft}
\DeclareSIUnit\dollar{\$}
\DeclareSIUnit\cent{\text{\textcent}}
\DeclareSIUnit\c{\degreeCelsius}

% Code Listing config
\usepackage{xcolor}
\definecolor{codegreen}{rgb}{0,0.6,0}
\definecolor{codegray}{rgb}{0.5,0.5,0.5}
\definecolor{codepurple}{rgb}{0.58,0,0.82}
\definecolor{backcolour}{rgb}{0.95,0.95,0.92}
\lstdefinestyle{overleaf}{
	% backgroundcolor=\color{backcolour},
	commentstyle=\color{codegreen},
	keywordstyle=\color{magenta},
	numberstyle=\tiny\color{codegray},
	stringstyle=\color{codepurple},
	basicstyle=\ttfamily\footnotesize,
	breakatwhitespace=false,
	breaklines=true,
	captionpos=b,
	keepspaces=true,
	numbers=left,
	numbersep=5pt,
	showspaces=false,
	showstringspaces=false,
	showtabs=false,
	tabsize=4
}

\usepackage[latte]{catppuccinpalette}
\lstdefinestyle{catppuccin}{
	breaklines=true,
	keepspaces=true,
	numbers=left,
	numbersep=5pt,
	showspaces=false,
	showstringspaces=false,
	breakatwhitespace=true,
	tabsize=4,
	stringstyle = {\color{CtpGreen}},
	commentstyle={\color{CtpOverlay1}},
	basicstyle = {\small\color{CtpText}\ttfamily},
	keywordstyle = {\color{CtpMauve}},
	keywordstyle = [2]{\color{CtpBlue}},
	keywordstyle = [3]{\color{CtpYellow}},
	keywordstyle = [4]{\color{CtpLavender}},
	keywordstyle = [5]{\color{CtpPeach}},
	keywordstyle = [6]{\color{CtpTeal}}
}

\lstset{style=catppuccin}


%
% Homework Details
%   - Title
%   - Subtitle
%   - Due date
%   - Due time
%   - Course
%   - Section/Time
%   - Instructor
%   - Author
%

\newcommand{\hmwkTitle}{Homework 02}
\newcommand{\hmwkSubTitle}{2BP}
\newcommand{\hmwkDueDate}{February 11th, 2025}
\newcommand{\hmwkDueTime}{11:59 PM}
\newcommand{\hmwkClass}{ENAE 404 - 0101}
\newcommand{\hmwkClassTime}{09:30}
\newcommand{\hmwkClassInstructor}{Dr. Barbee}
\newcommand{\hmwkAuthorName}{\textbf{Vai Srivastava}}
\newcommand{\hmwkCompletionDate}{\today}

\begin{document}

\maketitle

\pagebreak

\begin{hwkProblem}{1}{}

	Given the following position and velocity vectors, calculate the Keplerian orbital elements, assuming Earth is the central body. Do not use a computer code to do this. Vectors are in units of \unit{\km} and \unit{\km\per\s}.
	\begin{align*}
		\vec{r} = 3634.1 \bm{\hat{x}} + 5926 \bm{\hat{y}} + 1206.6 \bm{\hat{z}} \\
		\vec{v} = -6.9049 \bm{\hat{x}} + 4.3136 \bm{\hat{y}} + 2.6163 \bm{\hat{z}}
	\end{align*}

	\hwkSol

\end{hwkProblem}
\begin{hwkProblem}{2}{}

	\begin{enumerate}
		\item Write code to convert form Cartesian coordinates to orbital elements.
		\item Using subplot, plot the osculating orbital elements for the orbit of Didymos from HW00.
		\item Describe why your plots make sense (in reference to both the time variation of the orbital elements as well as the plot of the orbit in 3D space).
	\end{enumerate}

	\hwkSol

\end{hwkProblem}
\begin{hwkProblem}{3}{}
	
	Given the following orbit: \( a=\qty{2e4}{\km}, e=0.4, i=\qty{100}{\degree}, \Omega=\qty{30}{\degree}, \omega=\qty{15}{\degree}, \nu=\qty{15}{\degree} \)
	\begin{enumerate}
		\item Write code to convert from orbital elements to Cartesian coordinates.
		\item Propogate the orbit (around Earth) for one period.
		\item State the period of the orbit.
		\item Plot the orbit in 3D (use equal-length axes).
		\item Plot the deviation of the energy as compared to the inital energy \( \left( E_i - E_0 \right) \).
		\item Plot the osculating orbital elements.
	\end{enumerate}
	
	\hwkSol
	
\end{hwkProblem}
\begin{hwkProblem}{4}{}
	Sketch the following orbits in 2D and 3D. Assume that none of the spacecraft impact Earth.
	\begin{itemize}
		\item In the 2D orbit, label:
		\begin{itemize}
			\item periapsis
			\item angular momentum vector
			\item ascending node
			\item descending node
			\item spacecraft location
			\item portion of the orbit in the southern hemisphere
		\end{itemize}
		\item In the 3D orbit, label:
		\begin{itemize}
			\item angular momentum vector
			\item ascending node
			\item periapsis
		\end{itemize}
	\end{itemize}
	\begin{center}
		\begin{tabular}{cccccc}
			\hline
			Spacecraft ID & e & \(\mathrm{i} \left( \unit{\degree} \right) \) & \(\Omega \left( \unit{\degree} \right)\) & \(\omega \left( \unit{\degree} \right)\) & \(\nu \left( \unit{\degree} \right)\) \\
			\hline
			\hline
			A & 0.3 & 60 & 30 & 160 & 30 \\
			\hline
			B & 0.3 & 60 & 330 & 90 & 10 \\
			\hline
			C & 0.5 & 120 & 30 & 30 & 180 \\
			\hline
		\end{tabular}
	\end{center}
	
	\hwkSol

\end{hwkProblem}
\begin{hwkProblem}{5}{}
	
	Consider a spacecraft on a hyperbolic trajectory that will fly by Mars. The trajectory's semi-major axis is \qty{-11e3}{\km} and its eccentricity is \( 1.8 \). Calculate:
	\begin{enumerate}
		\item Turn angle
		\item Miss distance
		\item Hyperbolic excess speed
		\item Radius of periapsis of the flyby
	\end{enumerate}
	
	\hwkSol

	
\end{hwkProblem}
\end{document}
