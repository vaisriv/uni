\documentclass{article}

\usepackage{fancyhdr}
\usepackage{extramarks}
\usepackage{amsmath}
\usepackage{amsthm}
\usepackage{amsfonts}
\usepackage{amssymb}
\usepackage{xparse}
\usepackage{tikz}
\usepackage{graphicx}
\usepackage[plain]{algorithm}
\usepackage{algpseudocode}
\usepackage{listings}
\usepackage{hyperref}
\usepackage[per-mode = fraction]{siunitx}
\usepackage{calc}
\usepackage{cancel}

\usetikzlibrary{automata,positioning}

\hypersetup{
    colorlinks=true,
    linkcolor=blue,
    filecolor=magenta,
    urlcolor=blue,
    }

\urlstyle{same}

%
% Basic Document Settings
%

\topmargin=-0.45in
\evensidemargin=0in
\oddsidemargin=0in
\textwidth=6.5in
\textheight=9.0in
\headsep=0.25in

\linespread{1.1}

\pagestyle{fancy}
\lhead{\hmwkAuthorName}
\chead{\hmwkClass\ (\hmwkClassInstructor,\ \hmwkClassTime): \hmwkTitle}
\rhead{\firstxmark}
\lfoot{\lastxmark}
\cfoot{\thepage}

\renewcommand\headrulewidth{0.4pt}
\renewcommand\footrulewidth{0.4pt}

\setlength\parindent{0pt}
\allowdisplaybreaks
%
% Title Page
%

\title{
	\vspace{2in}
	\textmd{\textbf{\hmwkClass}}\\
	\textmd{\textbf{\hmwkTitle:}\ \hmwkSubTitle}\\
	\normalsize\vspace{0.1in}\small{Due\ on\ \hmwkDueDate\ at \hmwkDueTime}\\
	\vspace{0.1in}\large{\textit{\hmwkClassInstructor,\ \hmwkClassTime}}
	\vspace{3in}
}
\author{\textbf{\hmwkAuthorName}}
\date{\hmwkCompletionDate}

%
% Create Problem Sections
%

\newcommand{\enterProblemHeader}[1]{
	\nobreak\extramarks{}{Problem #1 continued on next page\ldots}\nobreak{}
	\nobreak\extramarks{Problem #1 (continued)}{Problem #1 continued on next page\ldots}\nobreak{}
}

\newcommand{\exitProblemHeader}[1]{
	\nobreak\extramarks{Problem #1 (continued)}{Problem #1 continued on next page\ldots}\nobreak{}
	\nobreak\extramarks{Problem #1}{}\nobreak{}
}

%
% Homework Problem Environment
%
\NewDocumentEnvironment{hwkProblem}{m m s}{
	\section*{Problem #1: #2}
	\enterProblemHeader{#1}
	\setcounter{partCounter}{1}
}{
	\exitProblemHeader{#1}
	\IfBooleanF{#3} % if star, no new page
		{\newpage}
}

% Alias for the Solution section header
\newcommand{\hwkSol}{\vspace{\baselineskip / 2}\textbf{\Large Solution}\vspace{\baselineskip / 2}}

% Alias for the Solution Part subsection header
\newcounter{partCounter}
\newcommand{\hwkPart}{
	\vspace{\baselineskip / 2}
	\textbf{\large Part \Alph{partCounter}}
	\vspace{\baselineskip / 2}
	\stepcounter{partCounter}
}

%
% Various Helper Commands
%

% Such That
\newcommand{\st}{\text{s.t.}}

% Useful for algorithms
\newcommand{\alg}[1]{\textsc{\bfseries \footnotesize #1}}

% For derivatives
\newcommand{\deriv}[1]{\frac{\mathrm{d}}{\mathrm{d}x} (#1)}

% For partial derivatives
\newcommand{\pderiv}[2]{\frac{\partial}{\partial #1} (#2)}

% Integral dx
\newcommand{\dx}{\mathrm{d}x}
\newcommand{\dy}{\mathrm{d}y}

% Probability commands: Expectation, Variance, Covariance, Bias
\newcommand{\e}[1]{\mathrm{e}#1}
\newcommand{\E}{\mathrm{E}}
\newcommand{\Var}{\mathrm{Var}}
\newcommand{\Cov}{\mathrm{Cov}}
\newcommand{\Bias}{\mathrm{Bias}}

% Col and Row Vectors
\newcommand{\crvector}[1]{\ensuremath{\begin{pmatrix}#1\end{pmatrix}}}

% Defining Units that are not in the SI base
\DeclareSIUnit\bar{bar}
\DeclareSIUnit\ft{ft}
\DeclareSIUnit\dollar{\$}
\DeclareSIUnit\cent{\text{\textcent}}
\DeclareSIUnit\c{\degreeCelsius}

% Code Listing config
\usepackage{xcolor}
\definecolor{codegreen}{rgb}{0,0.6,0}
\definecolor{codegray}{rgb}{0.5,0.5,0.5}
\definecolor{codepurple}{rgb}{0.58,0,0.82}
\definecolor{backcolour}{rgb}{0.95,0.95,0.92}
\lstdefinestyle{overleaf}{
	% backgroundcolor=\color{backcolour},
	commentstyle=\color{codegreen},
	keywordstyle=\color{magenta},
	numberstyle=\tiny\color{codegray},
	stringstyle=\color{codepurple},
	basicstyle=\ttfamily\footnotesize,
	breakatwhitespace=false,
	breaklines=true,
	captionpos=b,
	keepspaces=true,
	numbers=left,
	numbersep=5pt,
	showspaces=false,
	showstringspaces=false,
	showtabs=false,
	tabsize=4
}

% \usepackage[latte]{catppuccinpalette}
% \lstdefinestyle{catppuccin}{
% 	breaklines=true,
% 	keepspaces=true,
% 	numbers=left,
% 	numbersep=5pt,
% 	showspaces=false,
% 	showstringspaces=false,
% 	breakatwhitespace=true,
% 	tabsize=4,
% 	stringstyle = {\color{CtpGreen}},
% 	commentstyle={\color{CtpOverlay1}},
% 	basicstyle = {\small\color{CtpText}\ttfamily},
% 	keywordstyle = {\color{CtpMauve}},
% 	keywordstyle = [2]{\color{CtpBlue}},
% 	keywordstyle = [3]{\color{CtpYellow}},
% 	keywordstyle = [4]{\color{CtpLavender}},
% 	keywordstyle = [5]{\color{CtpPeach}},
% 	keywordstyle = [6]{\color{CtpTeal}}
% }

\lstset{style=overleaf}


%
% Homework Details
%   - Title
%   - Subtitle
%   - Due date
%   - Due time
%   - Course
%   - Section/Time
%   - Instructor
%   - Author
%

\newcommand{\hmwkTitle}{Extra Credit 01}
\newcommand{\hmwkSubTitle}{Technical Seminar}
\newcommand{\hmwkDueDate}{May 01st, 2025}
\newcommand{\hmwkDueTime}{09:30 AM}
\newcommand{\hmwkClass}{ENAE 404 - 0101}
\newcommand{\hmwkClassTime}{09:30 AM}
\newcommand{\hmwkClassInstructor}{Dr. Barbee}
\newcommand{\hmwkAuthorName}{\textbf{Vai Srivastava}}
\newcommand{\hmwkCompletionDate}{\today}

\begin{document}

\maketitle

\pagebreak

\section*{Astronomy Colloquium: Stellar Flares in a New Light}
\textsc{\large Dr. Rachel Osten (STSci) - April 23rd, 2025}

\subsection*{Seminar Summary}

Dr. Osten's seminar opened with a “Topic Preview” situating stellar flares on the cool half of the HR (Hertzsprung–Russell) diagram and emphasizing the rapid evolution of our understanding. Recent surveys show that flares are ubiquitous among late‐type stars and manifest a wide range of strengths and frequencies. Dr. Osten then highlighted that flares arise from magnetic reconnection events in the stellar corona, releasing energy impulsively into all atmospheric layers. These eruptive episodes not only produce bright X-ray and ultraviolet emission but also accelerate particles to MeV–GeV energies, forming characteristic loop-like structures and occasionally driving coronal mass ejections (CMEs).

A detailed discussion of flare physics followed. Dr. Osten reviewed how twisted magnetic field lines store energy until reconnection triggers a sudden release, heating plasma and driving shocks. Laboratory and space-based observations confirm that flares begin in the corona and propagate downward through the chromosphere and photosphere, engaging complex chemical and radiative processes. On the Sun, CMEs associated with powerful flares eject billions of tons of plasma, but scaling these processes to distant stars poses challenges due to the limited sensitivity of current instruments.

The seminar then turned to the “General Impacts on Exo‐Planets.” Stellar flares can strip planetary atmospheres via enhanced escape, erode protective ozone layers, and induce surface radiation hazards. Intriguingly, Dr. Osten noted emerging astrobiology models suggesting flare-driven UV flux and particle precipitation may catalyze the synthesis of prebiotic macro-molecules. While these chemical‐pathway hypotheses remain speculative, they offer a novel mechanism by which high‐energy stellar activity could contribute to the emergence of life.

Finally, the seminar contrasted solar and stellar flare studies. Solar flares benefit from in situ and multi‐wavelength measurements—nonthermal hard X-rays, microwaves, and high‐cadence UV imaging—whereas stellar flares must be inferred from more indirect proxies. Dr. Osten reviewed three principal techniques: Type II radio bursts (with drift rate $\dot{\nu} = \frac{\nu\cos(\theta\nu_B)}{2H_N}$), Doppler‐shift signatures of prominence eruptions, and “mass‐loss dimming” events indicating transient coronal holes. Each method has drawbacks—rare detections, ambiguous geometries, or unrealistic mass‐loss estimates when naively scaled from solar analogs. The talk concluded by underscoring the need for improved radio instrumentation and inclusion of UV/NUV observations to capture short (<10 s), millimeter‐wavelength flares that recent studies show are both frequent and energetically significant.

\subsection*{Relation to Course}

ENAE404 (Spaceflight Dynamics) covers the motion of spacecraft through environments shaped by gravitational, electromagnetic, and atmospheric forces. Understanding geo-effective aspects of solar activity—especially CMEs and stellar energetic particles (SEPs)—is critical for mission design, as sudden changes in space plasma conditions can perturb spacecraft trajectories, induce drag in low-Earth orbit, and impose radiation hazards on electronics and crew. The seminar’s treatment of magnetic reconnection and particle acceleration processes underscores the importance of plasma dynamics and electromagnetic field interactions, topics that complement classical orbit theory by highlighting non-gravitational perturbations. Moreover, the diagnostic techniques for remote flare observation mirror the kinds of sensor data interpretation and mission operations challenges students encounter when designing instrumentation for space weather monitoring.

\subsection*{Points of Interest}

I was particularly intrigued by the seminar’s “Geo-Effective Aspects of Flares” section, which examined how CMEs and SEPs influence planetary environments. Of special interest was the suggestion that flare-driven radiation might catalyze chemical pathways forming the macro-molecules essential for life. Although Dr. Osten cautioned that current astrobiology modules linking CME/SEP effects to prebiotic chemistry remain uncertain and may overestimate the true contribution, the concept opens an exciting line of inquiry: if high-energy stellar activity can indeed seed organic building blocks, it would reshape our understanding of life’s cosmic origins. Even if the hypothesis does not fully withstand future experimental tests, it offers a compelling research topic at the intersection of space weather, astrochemistry, astrobiology, and the search for life beyond Earth.

\end{document}
